%
% Documento: Introdução
%

\chapter{Introdução}\label{chap:introducao}
Este modelo de classe em \LaTeX \ de trabalho de final de curso foi elaborado para o programa de Pós-Graduação em Modelagem Computacional em Ciência e Tecnologia\footnote{Elaborado por Valdex Santos, discente do programa}, mas também pode ser utilizado por outros programas de graduação e pós-graduação da UESC. 

A classe {\ttfamily ppgmc-uesc.cls} \ foi construída com base nas normas da ABNT e da UESC para trabalhos de conclusão de curso. Maiores detalhes relacionados aos comandos existentes no estilo podrão da ABNT devem ser adquiridos através da documentação disponível no site \href{https://code.google.com/p/abntex2/}{https://code.google.com/p/abntex2/} \cite{abntex2classe}. Encorajamos a consultar as regras do pacote abntex em \href{http://ctan.tche.br/macros/latex/contrib/abntex2/doc/abntex2.pdf}{http://ctan.tche.br/macros/latex/contrib/abntex2/doc/abntex2.pdf}

Para melhor entendimento do uso do estilo de formatação, aconselha-se que o  usuário analise os comandos existentes no arquivo {\ttfamily PPGMC.tex} e os resultados obtidos no arquivo {\ttfamily PPGMC.pdf} depois do processamento pelo software LATEX + BIBTEX \cite{latex14,BibTeX14}.

Os arquivos pré-textuais, textuais e pós-textuais podem ser modificados a vontade. Nas pastas contém apenas exemplos. As referências citadas também são apenas exemplos, não contendo os textos que acompanham. O arquivo principal {\ttfamily PPGMC.tex} também pode ser modificado ou renomeado.

A utilização dessa classe reque conhecimento prévio de Latex. Esta é nossa primeira versão da classe, então sugestões de como melhorá-la e indicação de possíveis correções serão muito bem recebidas. A partir do segundo capítulo descrevemos como utilizar esta classe. Na seção \ref{sec:contatos} estão os contatos do autor para dúvidas e contribuições.

\section{Motivação}
\label{sec:motivacao}

Tendo em vista que o programa de Pós-Graduação em Modelagem Computacional em Ciência e Tecnologia ainda não tinha um estilo padrão para teses e dissertações, nem mesmo a UESC tinha esse padrão \LaTeX (pelo menos não foi encontrado pelo autor em nenhuma fonte), o autor a pedido do seu orientador \imprimirorientador \ resolveu desenvolver uma classe de padronização dos trabalhos do programa.

O estilo de documento utilizado é o {\ttfamily abntex2}. Através desse estilo a constituição do documento torna-se facilitada, uma vez que o mesmo possui comandos especiais para auxiliar a distribuição/definição das diversas partes constituintes do projeto. 


Uma das principais vantagens do uso do estilo de formatação para LATEX é a formatação \textit{automática} dos elementos que compõem um documento acadêmico, tais como capa, folha de rosto, dedicatória, agradecimentos, epígrafe, resumo, abstract, listas de figuras, tabelas, siglas e símbolos, sumário, capítulos, referências, etc.
