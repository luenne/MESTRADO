%
% Documento: Introdução
%

\chapter{Introdução}\label{chap:introducao}
Aglomerados de galáxias são as maiores estruturas do Universo observável que podem ter alcançado o estado de equilíbrio dinâmico. Eles são constituídos por algumas dezenas até milhares de galáxias ligadas pela força gravitacional. Para que as galáxias se mantenham próximas umas das outras por longas escalas de tempo é necessário que exista uma considerável força gravitacional que impeça a sua dispersão no espaço. Isto significa que a massa típica desses sistemas é muito grande (o catálogo de Abell, por exemplo, possui sistemas com massa total por volta de $10^{14}$ a $10^{15}$ massas solares\footnote{1 massa solar ${M_\odot}$ equivale a $2\times 10^{30}$ kg.}), o que torna o mapeamento de aglomerados e a correta obtenção de suas propriedades observadas de grande importância tanto para estudos referentes ao processo de formação de estruturas no Universo, como para restringir o modelo cosmológico atual (vide, por exemplo, \citeonline{VELASQUEZ2012}).

Ao longo do século XX muitos trabalhos foram voltados ao estudo de aglomerados e permitiram a sua crescente caracterização como um sistema físico bastante particular. Dessa maneira, propriedades como: a distribuição de posições e de velocidades em aglomerados, perfis de densidade numérica – entendimento quanto à distribuição das galáxias em relação a sua distância ao centro do aglomerado –, funções de luminosidade – quantificar a distribuição luminosa – e de massa, puderam ser obtidas e utilizadas para descrever aglomerados \cite{VELASQUEZ2012}. Além de parâmetros dinâmicos e cinemáticos como: velocidade média (adequado a estimativa de distâncias); dispersão de velocidades (compreensão do grau de ligação gravitacional entre galáxias); massas (entender o grau de contribuição de densidade referente a massa total do Universo); e a razão massa/luminosidade (utilizada como indicativo da quantidade matéria escura e para estimar se a distribuição de luz segue a da matéria) \cite{friaca2008}

As conclusões alcançadas por esses trabalhos sugerem que os aglomerados iniciaram o seu processo de formação há aproximadamente 10 bilhões de anos, processo que se dá de forma continuada até os dias de hoje. Estudos baseados na distribuição de velocidades de galáxias nesses sistemas indicam que apenas uma fração deles (em torno de 60\%) pode ser considerada em estado de equilíbrio dinâmico \cite{RAQUEL2012}. Os demais constituem sistemas ainda em formação ou perturbados por interações com outros aglomerados. O grande número de sistemas fora do equilíbrio pode introduzir dificuldades na interpretação das propriedades dinâmicas dos aglomerados \cite{friaca2008}.

Portanto, o entendimento preciso dos graus de liberdade de aglomerados é de extrema importância para que seja possível realizar inferências dinâmicas sobre esses sistemas. Um dos aspectos menos estudados a respeito de aglomerados é a possibilidade de que eles tenham movimento de rotação. Ao calcular a massa do aglomerado usando as velocidades individuais das galáxias membro admite-se que o aglomerado está em equilíbrio virial, desprovido de rotação, e seu potencial gravitacional é duas vezes a soma da energia cinética dos membros e as órbitas das galáxias são aproximadamente isotrópicas. Não levar em consideração uma possível rotação de aglomerados pode gerar um erro nas suas estimativas de massa, o que afeta diretamente as restrições cosmológicas fornecidas pela função de massa desses sistemas \cite{fang2009rotation}.

O ponto central do presente trabalho é desenvolver uma metodologia para identificar a componente rotacional dos aglomerados e corrigir, a partir da velocidade rotacional encontrada, a sua estimativa de massa. Com este propósito, implementamos na linguagem R o método proposto por Nascimento et.al (2016) adaptando-o para o estudo de rotação de aglomerados individuais. O trabalho está organizado da seguinte forma: no Capítulo 2 descrevemos o problema da rotação de aglomerados de galáxias, com uma breve definição de aglomerados de galáxias e sua distribuição de velocidades; no Capítulo 3 apresentamos os conceitos e ferramentas estatísticas utilizadas no projeto além da linguagem R e seus respectivos pacotes; Capítulo 4 fazemos uma descrição dos catálogos utilizados, assim como introduzimos o nosso método e o método de Hwang \& Lee; no Capítulo 5 apresentamos os resultados da nossa análise, bem como a correção das massas dos aglomerados; finalmente, no Capítulo 6 fazemos nossas considerações finais e discutimos alguns desdobramentos possíveis deste trabalho.     

