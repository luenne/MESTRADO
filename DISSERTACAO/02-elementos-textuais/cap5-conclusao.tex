%
% Documento: Conclusão
\chapter{Conclusão}\label{chap:conclusao}
Neste trabalho apresentamos um método de detecção de rotação em aglomerados de galáxias desenvolvido em R. Nosso objetivo foi investigar um dos aspectos menos estudados a respeito de aglomerados, que é a possibilidade de que eles tenham algum grau de rotação. Não levar em consideração a rotação de aglomerados pode gerar um erro nas suas estimativas de massa. O entendimento, controle e redução deste tipo de erro são de grande importância para a astrofísica extragaláctica. 

Utilizamos os catálogos \textbf{selec20}, composto por 20 aglomerados; \textbf{NoSocs}, com 183 aglomerados, sendo que 25 aglomerados continham um total inferior a 20 objetos, sendo inviável o cálculo de detecção; \textbf{III} que geramos duas amostras I e II de aglomerados com grau de rotação e sem rotação, respectivamente. 

Além da aplicação do nosso método nos catálogos, implementamos o método de Hwang \& Lee que utiliza a relação sinoidal para o cálculo do ângulo do eixo de rotação $\theta_o$ e a velocidade de rotação $v_{rot}$. Os resultados foram comparados e como conclusão temos que:

\begin{itemize}
   \item \textbf{Para o catálogo I (selec20)}: o nosso método detectou 14 aglomerados com evidência de rotação. Já o método de Hwang \& Lee detectou um grau de rotação para todos os aglomerados. 
   \item \textbf{Para o catálogo II (NoSocs)}: com total de 158 aglomerados, aproximadamente 39.87\%, ou seja, 63 dos aglomerados, dectectaram rotação. Diferente do nosso método, o de Hwang \& Lee considerou os 183 aglomerados e detectou rotação em 97.81\% deles.   
   \item \textbf{Para o catálogo I (selec20)}: para este catálogo simulamos duas amostras I e II com 200 aglomerados em cada um deles. Na amostra I todos os aglomerados tinham um grau de rotação. Já na amostra II, todos os aglomerados eram sem rotação. No nosso método, para a amostra I, detectamos rotação em todos os aglomerados, enquanto o método de Hwang \& Lee apenas 96.5\% dos casos. Para a amostra II, o nosso método detectou rotação em 13.5\% dos aglomerados e o de Hwang \& Lee um total de 96.5\%.
 \end{itemize}     

 Como desconhecemos as propriedades dinâmicas dos catálogos I e II utilizamos o método de Monte Carlo para simular amostras de aglomerados com rotação e sem rotação. Nesse cenário, nosso método obteve resultados mais seguros.


\section{Trabalhos Futuros}
\label{sec:trabalhosFuturos}

Como perspectivas, temos que uma possível continuação deste trabalho é realizar a correção no cálculo da massa através da fórmula \ref{eq:massa}. 

\begin{equation}
\omega^2 R = {GM}/{R},
\label{eq:massa}
\end{equation}
onde $\omega$ é a velocidade rotacional, R é o raio, M a massa e G é a constante gravitacional.