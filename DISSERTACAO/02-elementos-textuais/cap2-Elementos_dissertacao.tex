
\chapter{Rotação de Aglomerados de Galáxias}
\section{Aglomerado de Galáxias}
A origem do Universo, de acordo ao modelo cosmológico padrão, se deu há aproximadamente 14 milhões de anos. Desde então o seu processo de expansão ocorre de forma contínua e hierárquica, de modo que unidades menores se fundem formando outras maiores. Aglomerados de galáxias são as maiores estruturas do Universo observável e compõem os objetos de estudo desta dissertação \cite{ryden2017introduction}. 

Aglomerados de galáxias são definidos basicamente por três componentes: galáxias, meio intra-aglomerado e matéria escura. A maior parte da massa do aglomerado, cerca de 80\% do total, é composta de matéria escura (não-bariônica). Do restante, na forma bariônica (feita de prótons e nêutrons), 15\% são compreendidos de gás intra-aglomerado (MIA) e apenas 5\% da massa de um aglomerado estão na forma de estrelas e galáxias.

A busca por compreender a formação e evolução dos aglomerados de galáxias é uma das questões mais importantes da Astrofísica. No paradigma atual de formação das estruturas, as galáxias e os aglomerados surgem a partir de halos escuros. O resfriamento desses halos ocasiona a formação de estruturas condensadas, onde depois colapsariam os bárions (prótons e nêutrons), formando  os sistemas astrofísicos conhecidos. Este cenário seria ainda hierárquico, com a formação dos aglomerados ocorrendo após a formação das galáxias e pequenos grupos, aproximadamente em um desvio para o vermelho $z \approx 2$ (vide, por exemplo,\citeonline{VELASQUEZ2012}).

\begin{figure}[H]
	\centering
	\includegraphics[width=0.5\textwidth]{04-figuras/10043dist}
	\caption{Histograma de velocidades de um dos aglomerados de nossa amostra.}
	\label{fig1}
\end{figure}

O processo de formação de aglomerados de galáxias ainda não atingiu o seu fim. Enquanto regiões centrais estão em equilíbrio dinâmico, as regiões periféricas (externas) acumulam matéria na forma de galáxias ou grupos de galáxias de modo contínuo. Comumente o entorno dos aglomerados de galáxias é constituído de grupos de galáxias que podem ser absorvidos pelo aglomerado principal ao longo do tempo, ocasionando o aumento de sua massa \cite{rembold2011}. Estudos sobre a distribuição de velocidades de galáxias em aglomerados indicam que a mesma possui distribuição gaussiana, vide Figura \ref{fig1}, ou muito bem ajustada por uma gaussiana somente na região virializada do sistema (região mais interna do sistema) \cite{yahil1977velocity}, podendo existir sinais de múltiplos modos normais na região mais externa \cite{ribeiro2011non}, comprovando a presença de componentes de um sistema em processo de evolução pelo acréscimo de matéria ao seu entorno. Esse acréscimo de matéria, na forma de galáxias ou grupos de galáxias.  Isto sugere que a formação de aglomerados de galáxias é um processo contínuo que decorre de fusões sucessivas e encontros gravitacionais de maiores e menores proporções \cite{nascimento2016dynamical}.


\section{Distribuição de Velocidades ao longo do Aglomerado}
A velocidade de uma galáxia contida em um aglomerado, em uma dada posição, não pode ser maior que a velocidade de escape do sistema, caso isto aconteça a galáxia não pertenceria mais ao aglomerado. A velocidade de escape e a distância ao centro do aglomerado são grandezas inversamente proporcionais, ou seja, a velocidade de escape decresce com o aumento da distância ao centro do aglomerado, portanto é mais fácil o escape de uma galáxia que está na região periférica do sistema \cite{viegas2004}.

Para que o aglomerado exista como unidade dinâmica é preciso uma redução na amplitude da distribuição de velocidades das galáxias à medida que haja um afastamento da região central. O grande problema dessa propriedade é o efeito de projeção. Galáxias que estão com distâncias distintas do centro do aglomerado podem parecer ao observador com mesma distância em consequência de a observação referir-se apenas às posições projetadas no plano do céu \cite{viegas2004}. Ainda assim é possível estimar com qualidade suficiente os membros dos aglomerados usando-se o perfil da velocidade 
em função do raio do aglomerado.

Na Figura \ref{fig2} vemos a distribuição de velocidades do aglomerado em função da distância da galáxia ao centro do aglomerado, onde o estreitamento da distribuição de velocidades define uma espécie de "corneta" que pode ser utilizada para definir os membros de um aglomerado, sendo removidas as galáxias que estejam significativamente acima ou abaixo da "corneta". O contorno da corneta pode ser definido de várias formas, mas basicamente
representa a velocidade limite em cada raio para que uma galáxia esteja ligada gravitacionalmente ao sistema.

São os membros dos aglaomerados que definem a sua dinâmica, ou seja,
a estes objetos é que aplicamos o Teorema do Virial para obter propriedades importantes como dispersão de velocidades, massa e raio desses sistemas.
A dispersão de velocidades está diretamente ligada à massa do aglomerado, mas seu valor estimado não leva em conta
uma possível componente de velocidade referente à rotação global do aglomerado, o que pode introduzir um viés no cálculo da massa,
e revela a necessidade de desenvolvermos métodos para acessar este possível movimento do aglomerado.

\begin{figure}[H]
	\centering
	\includegraphics[width=0.5\textwidth]{04-figuras/10043}
	\caption{Distribuição de velocidades em função da distância ao centro de um dos aglomerados de nossa amostra.}
	\label{fig2}
\end{figure}

\section{Rotação de Aglomerados}

O conhecimento do estado dinâmico de aglomerados de galáxias pode propiciar restrições importantes em cenários cosmológicos, como a determinação da massa total do aglomerado e uma estimativa da quantidade de matéria escura no Universo. A possibilidade da existência de aglomerados em rotação tem sido discutida por muitos autores (por exemplo, veja os estudos de \cite{hwang2007searching, manolopoulou2016galaxy}). 

Para detecção de indícios de rotação, \citeonline{hwang2007searching} utilizaram dados espectroscópicos do \textit{Sloan Digital Sky Survey} \footnote{Considerado o mais ambicioso mapeamento astronômico que já foi feito. Com este mapeamento, os astrônomos podem observar os padrões de grande escala das galáxias: filamentos e vazios em grandes regiões angulares do Universo.}(SDSS) e \textit{Two-Degree-Field Galaxy Redshift Survey} (2dF-GRS). A rotação de aglomerados foi modelada globalmente, envolvendo tanto as galáxias membro como o gás intra-aglomerado. Eles levantaram inicialmente a hipótese de que a rotação se origina através de fusões de aglomerados. Um aspecto importante do método empregado por \citeonline{hwang2007searching} é que os aglomerados com rotação devem exibir divisão espacial entre galáxias com velocidades maiores e menores que a velocidade média do aglomerado, além de apresentar um pico no mapa de densidade. Nesta pesquisa de \citeonline{hwang2007searching} foram detectados seis sistemas com rotação, em um total de dozes aglomerados (Abell 0954, Abell 1139, Abell 1399, Abell 2162, Abell 2169, and Abell 2366). Constatou-se ainda que estes aglomerados estão em equilíbrio dinâmico e não sofreram fusão recente, não dando suporte, portanto, à hipótese de interações como causadoras da rotação. 
	
\citeonline{kalinkov2005rotation} tentaram obter o gradiente máximo no campo de velocidades de Abell 2107 e determinaram que a direção do coeficiente de correlação linear máximo definiria o eixo maior do aglomerado e o eixo menor seria o de rotação. Foram utilizadas subamostras de galáxias membro, ordenadas de acordo a distância ao centro do aglomerado para definir o grau de rotação do sistema. Esse mesmo aglomerado foi estudado por \citeonline{oegerle1992structure} e foram encontrados indícios de rotação. \citeonline{materne1983cluster} apontaram a dificuldade em diferenciar um aglomerado rotativo de dois que se sobrepõem, pelo motivo de estar se fundindo ou se afastando. Porém, o aglomerado  Abell 2107 não consiste de dois aglomerados sobrepostos, em consequência do pico estreito representado em seu histograma de velocidades. O indicador mais forte que definiu sua rotação foi o ângulo de posição do eixo com o gradiente máximo no campo de velocidades quase coincidindo com o ângulo de posição do eixo com o maior alongamento.  Nesse estudo o período de rotação foi calculado em uma volta a cada $2.4\times10^9$ anos e houve uma correção no valor da massa para $2.8\times10^{14}~{M_\odot}$  (a massa inicial era de $3.2\times10^{14}~{M_\odot}$, sem levar em conta a rotação).  Mais recentemente, Song et al. (2018) estudaram 285 galáxias membro em A2107, a partir de 
levantamento espectroscópico próprio, e encontraram forte sinal de rotação do aglomerado, pelo menos dentro 20 minutos de arco do seu centro,
embora com o sinal torne-se mais fraco para regiões externas. Estes autores também encontram conexões deste aglomerado com
estruturas filamentares em sua vizinhança que poderiam estar induzindo a rotação do sistema. Por outro lado,
Liu and Tozzi (2019) estudaram a distribuição de gás emissor de raios-X em A2107 usando dados do telescópio Chandra, 
mas não encontraram resultado conclusivo a respeito da rotação do sistema. \footnote{O aglomerado A2107 é um dos mais intrigantes
sistemas no universo próximo, contudo seu mapeamento feito pelo SDSS é bastante incompleto em espectroscopia (em dois
graus quadrados em torno de seu centro, encontramos apenas 78 galáxias membro) o que impede que o mesmo seja estudado neste trabalho.} \cite{liu2019testing, song2018redshift}

Baseado no estudo da distribuição de velocidades das galáxias membro, \citeonline{tovmassian2015rotation} detectou
sinais de rotação em 17 sistemas de uma amostra de 65 aglomerados (26\%).
O método analisa o número de galáxias com velocidades mais baixas e mais altas que a velocidade média do aglomerado em diferentes partes do aglomerado. O método teve mais êxito em aglomerados planos, com $f=a/b > 1.8$ (a e b semieixos - maior e menor – da distribuição de galáxias do aglomerado). 
Para estes, a taxa de detecção de rotação foi mais alta (7 dos 18 aglomerados planos, 39\%).  Esse resultado suporta a opinião de que os aglomerados foram originalmente formados a partir das enormes nuvens de gás primordiais e preservaram a rotação das nuvens primordiais, a menos que sofram fusões com outros aglomerados e grupos de galáxias. 

Já na tese de \citeonline{manolopoulou2016galaxy} é realizado um estudo de um novo algoritmo para dedução de rotação usando a velocidade radial projetada\footnote{É a componente da velocidade de um objeto na direção da linha de visada, isto é, a velocidade com que o objeto se aproxima ou se afasta do observador.}. Inicialmente os testes foram realizados em aglomerados gerados em simulações de Monte Carlo para confirmar se o método fornecia indicações robustas de rotação. Em seguida, aplicado em amostras de aglomerados de Abell. Através do teste estatístico de Kolmogorov-Smirnov, decidiu-se quanto a sua rotação significativa ou não, seu centro rotacional, orientação do eixo de rotação, amplitude de velocidade rotacional e, finalmente, o sentido de rotação no sentido horário ou anti-horário no plano do céu. Foram encontrados 23 aglomerados possivelmente rotativos dentro de 1.5 Mpc ou a uma distância de 2.5 Mpc do centro do aglomerado, do total de 45 da amostra.

\citeonline{nascimento2016dynamical}, a partir de uma amostra de galáxias observadas no Cerro Tololo Interamerican Observatory (CTIO), realizaram um estudo dinâmico em torno do par de aglomerados de Abell (A3407 e A3408). O objetivo era verificar se a amostra correspondia a um simples sistema de galáxias ou a um processo de fusão, melhorando o entendimento desse sistema. Testes estatísticos foram aplicados aos membros mostrando que ambos os sistemas bem como cada aglomerado individual tem uma distribuição de velocidade Gaussiana. Um gradiente de velocidades de $\approx 847 \pm 114\; {km~s^{-1}}$ foi identificado ao redor do eixo principal da distribuição de galáxias projetada indicando uma possível rotação. 
O estudo definiu uma lacuna (ou "gap") na distribuição de velocidades e realizou testes sobre a distribuição espacial de galáxias (em torno do eixo principal do aglomerado) visando identificar diferenças entre objetos com velocidades maiores e menores que a posição do "gap". Esta comparação indicou que havia diferença significativa entre estas subamostras, sugerindo um grau de rotação no sistema A3407+A3408 (vide Figura \ref{fig3}).

\begin{figure}[H] %h or !htbp
\vspace{-2pt}
\begin{center}
\includegraphics[height=9cm,width=9cm]{04-figuras/nascimento.png}%
\caption{Distribuição de galáxias no plano do céu do par de aglomerados A3407 e A3408.}
 \textbf{Fonte: Nascimento et al. 2016.}
\label{fig3}%
\end{center}
\end{figure}

\chapter{Ferramentas Estatísticas e Computacionais}

Durante a investigação que realizamos sobre o movimento de rotação de aglomerados, diversas
ferramentas estatísticas e computacionais foram utilizadas. Neste capítulo, apresentamos uma
breve descrição deste aparato estatístico-computacional.

\section{Testes de Hipótese}

Uma vez que nosso estudo fará uso de alguns testes estatísticos, fazemos aqui uma breve revisão destas importantes ferramentas da estatística inferencial.
De modo geral, a análise estatística objetiva visa fazer inferências sobre uma população a partir da observação de uma amostra, e os testes de hipótese representam uma forma de inferência estatística. A hipótese é uma afirmação sobre parâmetros populacionais que devem ser analisadas para verificar sua veracidade. É importante ressaltar que a "verdade" não  pode nunca ser determinada, a menos que toda a população seja observada, situação impraticável na maioria das vezes, justificado pelo uso do teste estatístico sobre amostras.

A princípio é necessário estabelecer a \textbf{hipótese nula}, denotada por \textbf{$H_0$}. Já a \textbf{hipótese alternativa ($H_1$)}, contrapõe a hipótese nula, ou seja, $H_0$ deverá ser rejeitada ou não em relação a $H_1$. É necessário estabelecer um critério auxiliar para decidir a rejeição ou não de $H_0$ para um teste estatístico. Esse valor, determinado pelo pesquisador antes da análise de dados ou até mesmo na coleta de dados, cenário ideal, é denominado \textbf{nível alfa ($\alpha$) ou nível de significância}. Comumente é utilizado como critério de rejeição uma significância de 5\%. De acordo com \citeonline{cramer2004sage}, 

\begin{citacao}
O nível em que a hipótese nula é rejeitada é geralmente definido como 5 ou menos vezes em 100. Isso significa admitir que um dado resultado ocorra por acaso 5 ou menos vezes de 100.  O nível de significância de 5\% foi historicamente uma escolha arbitrária, mas tem sido considerado uma escolha razoável na maioria das circunstâncias. Se houver um motivo para variar este nível, o pesquisador pode  fazê-lo. Então, em circunstâncias em que pode haver consequências adversas muito graves se a decisão errada foi feita sobre a hipótese, então o nível de significância poderia ser mais rigoroso em, digamos, 1\% \cite[p.~151]{cramer2004sage}. 
\end{citacao}

Na realização de testes de hipóteses é possível que erros sejam cometidos, como mostrado no quadro \ref{qua:hipotese}. \textbf{Erro do tipo I}, denotado por \textbf{erro $\alpha$}, a rejeição de $H_0$ quando ela é verdadeira. Contrapondo, a não-rejeição de $H_0$ quando esta é falsa é denominada \textbf{erro do tipo II} e representado por \textbf{$\beta$}.  Esse tipo de teste permite concluir se deve aceitar ou rejeitar a hipótese nula, porém não é possível quantificar o quão provável é o resultado de ocorrer ao acaso. Por esta razão é definida a potência de um teste estatístico $1-\beta$ como a probabilidade de rejeitar $H_0$ quando de fato é falsa. Claramente, o teste ideal é aquele em que os valores de $\alpha$ e $\beta$ são mínimos. Porém, o valor de $\alpha$ é inversamente relacionado com o valor de $\beta$, sendo impossível minimizá-los simultaneamente. Geralmente, é fixado o nível de significância $\alpha$ e escolhida a região de rejeição que minimiza $\beta$, ou seja, que maximize a potência do teste. 


\begin{quadro}[!htb]
	\centering
	\caption{Tipos de erros em testes de hipótese.\label{qua:hipotese}}
	\begin{tabular}{|l|l|l|ll}
		\cline{1-3}
\multicolumn{1}{|c|}{\multirow{2}{*}{\textbf{Decisão Estatística}}} & \multicolumn{2}{l|}{\textbf{Natureza (estado verdadeiro ou desconhecido)}} &  &  \\ \cline{2-3}
\multicolumn{1}{|c|}{} & \multicolumn{1}{c|}{\textbf{$H_0$ verdadeira}} & \multicolumn{1}{c|}{\textbf{$H_1$ falsa}} &  &  \\ \cline{1-3}
\textbf{Aceitar $H_0$} & acerto & Erro tipo II ( \ensuremath{\beta}) &  &  \\ \cline{1-3}
\textbf{Rejeitar $H_0$} & Erro tipo I ( \ensuremath{alpha}) & acerto &  &  \\ \cline{1-3}
	\end{tabular}
	\textbf{\fonte{\cite{lieber1990statistical} adaptada}}
\end{quadro}

O menor nível de significância pode ser definido utilizando o \textbf{valor-p} ou \textbf{\textit{“p-value”}}. No teste de hipótese esse valor é comparado ao nível de significância \ensuremath{alpha}, fixado no início, com o objetivo de decidir aceitar ou rejeitar $H_0$. Se o valor-p calculado do teste for igual ou maior que \ensuremath{alpha}, a $H_0$ é aceita. Ou seja, a hipótese nula é consistente com os resultados da amostra. Porém, se o valor-p for menor que \ensuremath{alpha}, a hipótese nula é rejeitada, e a hipótese alternativa, nesse caso, é então aceita como verdadeira.

\subsection{Teste de Normalidade}
Uma variável aleatória, por exemplo a idade de um grupo de pessoas ou as velocidades de galáxias em aglomerados, pode admitir uma distribuição de frequências da população, contendo diversas formas encontradas na literatura estatística. O intuito desses modelos é caracterizar o comportamento de um determinado evento em função da frequência de sua ocorrência. Se as variáveis forem contínuas, o evento será um intervalo de valores. Portanto, as distribuições de frequências são efetivamente distribuições de probabilidade, em que para um evento teremos associado uma probabilidade de ocorrência \cite{leotti2012normalidade}. 

A inspeção visual pode ser utilizada para avaliação de uma distribuição. A distribuição de frequência, usualmente apresentada em forma de histograma, relaciona valores observados à sua frequência. Pode-se, além disso, pressupor uma forma para a distribuição, identificar a ocorrência de lacunas nos dados, assim como a presença ou não de $outliers$ (valores atípicos em relação à distribuição). O histograma é composto por barras justapostas  que no eixo horizontal contêm a variável de interesse  e no eixo vertical a sua correspondente frequência \cite{leotti2012normalidade}. Para distribuições do tipo normal ou Gaussiana, o histograma constitui um formato de sino.

Entretanto, a simples constatação por meio de gráficos é subjetiva e não satisfatória, pois depende de uma interpretação visual que introduz  subjetividade na pesquisa. Desta forma, para inferir sobre a normalidade de uma distribuição uma abordagem possível é utilizar  testes estatísticos \cite{cirillo2003extensao}.  Como exemplo, podemos citar: os testes de aderência qui-quadrado; Kolmogorov-Smirnov; Anderson-Darling; Lilliefors e Shapiro-Wilk. 

Estes testes possuem estatísticas de teste e critérios de decisão diferentes, porém compartilham da hipótese avaliada: a hipótese nula ($H_0$) especifica que a variável aleatória adere à distribuição normal, sem a necessidade de definir a média ou variância da distribuição. Já a hipótese alternativa ($H_1$), opõe a hipótese nula, representando qualquer outra distribuição para os dados \cite{leotti2012normalidade}.

O resultado que interessa após executar um determinado teste é o seu valor-p ou nível descritivo do teste, referente à probabilidade de que a estatística do teste (como variável aleatória) tenha valor extremo, em comparação ao valor observado da estatística do teste, quando a hipótese nula é verdadeira. Sendo o valor-p menor que o nível de significância, a hipótese nula é rejeitada. Ou seja, o valor-p representa o menor nível de significância que se pode assumir para rejeitar a hipótese nula. Logo, há significância estatística quando o valor-p é menor que o nível de significância estabelecido \cite{FLAVIO2012}. 
 
Neste trabalho, utilizaremos o teste de normalidade de Anderson-Darling para relacionar a possível rotação de aglomerados com o
seu estado dinâmico global, usualmente indicado pela distribuição normal das velocidades de galáxias membro dentro
da região virial dos aglomerados.

\subsection{Testes de duas amostras}

Um outro recurso estatístico importante a ser utilizado neste trabalho é o teste de duas amostras, que visa
comparar dois subconjuntos de uma dada população.
O teste de duas amostras é uma análise estatística projetada para utilizar dados de duas amostras aleatórias. O intuito do teste é determinar se a diferença entre estatísticas de duas amostras (por exemplo, suas médias) seja compatível com que elas tenham sido extraídas de uma mesma população.

Sejam $X_1, X_2, ..., X_m$ e $Y_1, Y_2, ..., Y_N$ duas amostras aleatórias independentes com funções de distribuição contínua F e G, respectivamente, o teste de duas amostras verifica se

\begin{equation}
 H_0 : F = G  \hspace{2cm} H_1 : F \neq G
\label{eq:twoSample}
\end{equation}

Isto é, assume-se como hipótese nula que as médias populacionais das amostras não são significativamente diferentes.
Os testes de duas amostras podem ser univariados ou multivariados. Neste trabalho, nosso interesse 
está em testes 2D (bivariados) sobre a distribuição espacial de galáxias com determinadas características cinemáticas.

\subsection{Testes utilizados}
\subsubsection{Teste de Cramer-von Mises}
O teste de Cramer (conhecido também como phi de Cramer - $\varphi$c) é uma medida de associação entre duas variáveis nominais dado o intervalo de 0 a 1, indicando que um valor mais alto possui forte associação. Fundamentado no teste estatístico do qui-quadrado de Pearson, foi publicado em 1946 por Harald Cramer. A medida é definida como

\begin{equation}
\ V = \sqrt{\frac{{\chi_{obt}}^2}{N.m}}
\label{eq:eq1}
\end{equation}
	
	onde $\chi^2$ é o valor obtido do teste estatístico
	
	N é o tamanho da amostra e 
    
    m = o menor de (r - 1) ou (c – 1), sendo r o número de linhas e c o número de colunas.

Para entender melhor a utilidade do teste de Cramer é fundamental compreender as formas como os testes estatísticos divergem das medidas de associação para variáveis categóricas. O teste qui-quadrado ($\chi^2$) fornece um teste estatístico de associação entre duas variáveis categóricas (nominais) de uma população única. Ele determina se a associação entre as variáveis é significativa, utilizando como hipótese nula ($H_0$) que as duas variáveis não são dependentes uma da outra e como hipótese alternativa ($H_1$) que existe alguma associação entre duas variáveis \cite{bachman2017statistics}.

O teste de Cramer é considerado um dos mais utilizados entre as medidas baseadas no qui-quadrado. Geralmente, quando o seu cálculo resulta no valor máximo 1 é porque existe um forte relacionamento entre duas variáveis. No cálculo de Cramer é levado em consideração as dimensões da tabela, ou seja, diferentes dimensões podem ser comparadas significativamente \cite{bachman2017statistics}. 

\subsubsection{Teste de Hotelling}

Um dos mais conhecidos testes de hipóteses multivariados foi proposto por Harold Hotelling em 1947, o teste de $T^2$, compara vetores de médias populacionais. Baseado na generalização da estatística \textit{t de Student}, foi o primeiro a levar em consideração a correlação das variáveis na formulação da estatística do teste.

Sejam \textbf{\textit{X}} um vetor aleatório com uma dada dimensão, \textbf{\textit{$\mu$}} o vetor de médias e \textbf{\textit{$\sigma$}} a matriz de covariância. Para \textbf{\textit{X}}, sendo uma distribuição  multivariada e com tamanho de amostra aleatória \textbf{\textit{n}}, a estatística de $T^2$ é dada por
\begin{equation}
\ T^2 = n(\overline{X} - \mu_0) \sum_{pxp}^{-1} {(\overline{X} - \mu_0)}
\label{eq:eq2}
\end{equation}

com

\begin{equation}
 H_0 : \mu = \mu_0 \hspace{2cm} H_1 : \mu \neq \mu_0
\label{eq:eq3}
\end{equation}

A equação \ref{eq:eq2} tem distribuição qui-quadrado com \textit{p} graus de liberdade. Definindo um nível de significância $\alpha$, com $0 < \alpha < 1$, para valores de $T^2$  maiores ou iguais ao valor crítico ${\chi^2_{a,p,c}}$ dado por $P[{\chi_p}^2 \geq {\chi^2_{a,p,c}}]$,  a hipótese nula será rejeitada \cite{mudholkar2000robust}.    

Sendo a matriz desconhecida, a estatística de $T^2$ é dada por

\begin{equation}
T^2 = n(\overline{X} -\mu_0) S^{-1}(\overline{X} - \mu_0)
\label{eq:eq4}
\end{equation}

que sob a hipótese nula, tem uma distribuição proporcional a uma distribuição F (a distribuição F de Fisher-Snedecor), ou seja, o valor crítico do teste a um nível de significância $\alpha$, com $0 < \alpha < 1$, é

\begin{equation}
F_c = \frac{p(n-1)}{n-p} F_{1-\alpha, p, n - p}
\label{eq:eq5}
\end{equation}

onde $F_{1-\alpha, p, n - p}$ é a probabilidade acumulada igual a (1 - $\alpha$) da distribuição de F com p
n-p é igual a graus de liberdade.

Sendo S a matriz de covariâncias amostrais (\textit{pxp}), um estimado não viciado de $\sum_{pxp}$, dado por
\begin{equation}
\begin{bmatrix}
S^{2}_{1} & S_{12} & ... & S_{1p} \\ 
 & S^{2}_{2} & ... & S_{2p} \\ 
 &  & \ddots  & \vdots  \\ 
 &  &  & S^{2}_{p}
\end{bmatrix}
 \label{eq:eq6}
\end{equation}

em que os elementos da diagonal principal de S são as variâncias definidos por
\begin{equation}
S^2_j = \frac{1}{m-1} \sum_{k=1}^m (x_{jk} - \overline{X}_j),    j = 1, 2, ..., 3 
\label{eq:eq7}
\end{equation}

e os elementos fora da diagonal principal são as covariâncias conforme

\begin{equation}
S_{jh} = \frac{1}{m-1} \sum_{k=1}^m (x_{jk} - \overline{X}_j)(x_{hk} - \overline{X}_h) 
\label{eq:eq8}
\end{equation}

onde $x_{jk}$ e $x_{hk}$ representam os valores amostrais das variáveis $X_j$ e $X_h$.

\section{Linguagem R}

Neste trabalho, tanto os testes estatísticos descritos acima, assim como várias outras funções de variada utilidade, foram aplicados aos dados dentro do ambiente R.

R é uma linguagem e ambiente para análise estatística e produção de gráficos, desenvolvida na década de 90 pelos estatísticos Ross Ihaka e Robert Gentleman que utilizavam sistemas pagos em seus projetos. Contém diversos pacotes integrados, permitindo que sejam realizadas uma grande variedade de estatísticas (regressão linear, não-linear, séries temporais) e o uso de recursos gráficos avançados, o que a distingue das demais linguagens \cite{languageR}. 

O R é um conjunto integrado de instalações de software para manipulação de dados, cálculos e exibição gráfica. Isto abrange uma instalação eficaz de manipulação e armazenamento de dados, um conjunto de operadores para cálculos em matrizes, uma grande coleção coerente de ferramentas intermediárias para análise de dados e representações gráficas, a inclusão de condicionais (\textit{loops} e funções recursivas) definidas pelo usuário e recursos de entrada e saída \cite{languageR}.

\subsection{Pacotes utilizados neste trabalho}

\subsubsection{Pacote Cramer}
Uma rotina em R aplicada para o teste de Cramer de duas amostras. O valor de retorno é um objeto da classe \textit{"cramertest"}, contendo, dentre outros, os seguintes componentes:

\begin{itemize}
   \item \textit{\textbf{statistic}}: valor estatístico do teste de Cramer para observações.
   \item \textit{\textbf{conf.level}}: nível de significância do teste.
   \item \textit{\textbf{p.value}}: estimativa do valor-p.
 \end{itemize}  

\subsubsection{Pacote Hotelling}
Uma rotina em R aplicada para o teste de Hotelling de duas amostras. O valor de retorno é uma lista da classe \textit{"’hotelling.test"}, contendo, dentre outros, o seguinte componente:

\begin{itemize}
   \item \textit{\textbf{pval}}: estimativa do valor-p.
 \end{itemize} 

\subsubsection{Pacote nortest}

Conjunto de rotinas para testes de normalidade. Em particular, utilizamos o teste de Anderson-Darling do qual extraímos:


\begin{itemize}
   \item \textit{\textbf{pval}}: estimativa do valor-p.
 \end{itemize} 

\subsubsection{Pacote Ellipse}

Este pacote contém diversas rotinas para ajustar e desenhar elipses e regiões de confiança do tipo elipse, implementando os gráficos descritos por Murdoch e Chow (1996) e Bates e Watts (1988).

\subsubsection{Pacote Astro}

O pacote \textbf{astro} fornece uma série de funções, ferramentas e rotinas no uso diário da astronomia. Pode-se agrupar essas funções em áreas como a cosmologia \footnote{Funções que calculam distâncias, movimentação de volumes, \textit{lookback time} e luminosidade em uma cosmologia plana}, manipulação de arquivos FITS \footnote{Sistema flexível de transporte de imagens (\textit{Flexible Image Transport System}), um formato de arquivo comum em astronomia}, funções de tempo e posição. Entre outros.

\subsubsection{Pacote cosmoFns}
 O pacote contém expressões de distância, tempo, luminosidade e outras úteis na cosmologia observacional, compreendendo observações em linhas moleculares. Atualmente codificado apenas para o universo plano.

\chapter{Dados e Metodologia}

O método proposto por \citeonline{nascimento2016dynamical} para estudar um par de aglomerados, cujos membros podem estar em rotação em torno
do centro de massa do sistema, pode ser adaptado para o estudo da rotação de aglomerados individuais, onde deve-se estudar
possíveis correlações entre a coordenadas de posição com a de velocidade das galáxias de um dado sistema. Neste trabalho fazemos esta adaptação e a implementamos em linguagem R. O código resultante é aplicado a um conjunto de três catálogos: 

\begin{itemize}
   \item \textit{\textbf{Catálogo I: }} A amostra selec20, composta por 20 aglomerados ricos do SDSS, localizados em baixos $redshifts$ ($z \leq 0.13$), com espectroscopia disponível para objetos com $m_r \leq 17.77$.
   \item \textit{\textbf{Catálogo II: }} NoSOCS (\textit{\textbf{Northern Sky Optical Cluster Survey}}) também em  baixos $redshifts$ ($z \leq 0.10$), mas compreendendo um número maior de aglomerados (183 objetos).	
   \item \textit{\textbf{Catálogo III}}: Esferas de NFW \cite{NFW1997}, ou seja, modelos simulados de sistemas esféricos com ou sem rotação. Foram gerados 200 aglomerados com rotação e 200 sem rotação. O código que gera as esferas de NFW foi desenvolvido em linguagem fortran pelo Dr. Claudio Soriano Brandão, cedido para uso neste trabalho.
 \end{itemize}  

\section{Catálogo I: selec20}
A amostra selec20  corresponde a um conjunto de aglomerados ricos (sistemas com mais de 50 galáxias na região virializada) a baixos $redshifts$ (z < 0.13), cuja determinação de membros foi feita pelo Dr Paulo Lopes (UFRJ) fazendo uso do programa \textit{shifting gapper} (Lopes et al. 2009). O programa \textit{shifting gapper} remove galáxias intrusas (\textit{outliers}) e realiza a análise virial dos aglomerados, obtendo os valores de M200 e R200. A amostra selec20, por suas características, e por conter aglomerados bastante conhecidos do Universo Local, é  extremamente útil para testes e análises exploratórias. 


\section{Catálogo II: NoSOCS}
\label{subsec:simulate}

O NoSOCS (Gal et al. 2000, 2003, 2008) é um catálogo de aglomerados de galáxias elaborado a partir da versão digitalizada do Segundo Levantamento do Observatório de Palomar (POSS-II; DPOSS, Djorgovski et al. 2003, é a sua versão digital). Este catálogo é derivado de campos de alta latitude $\vert b \vert > 30^\circ$, cobrindo $ \sim 11,000  deg^2$ e contendo $\sim 15,500$ aglomerados candidatos. Sua construção é limitada para $r = 19.5$, onde a separação estrela/galáxia é confiável e erros fotométricos são precisos para usar a cor g-r como indicador do \textit{redshift}. A classificação de objetos e a calibração fotométrica são descritas em Odewahn et al. (2004) e Gal et al. (2004), respectivamente.

Os dados foram extraídos do SDSS para cada aglomerado do NoSOCS (Lopes 2003; \cite{lopes2004northern} amostrados no DR5 ($Data Release 5$). Com o uso de dados fotométricos de alta qualidade, foi possível estimar os novos \textit{redshifts} fotométricos (seguindo Lopes 2007), riqueza e luminosidade óptica \cite{lopes2006x}. Para esses objetos foi aplicado uma técnica de eliminação de interferentes de galáxias (ou seja, objetos em projeção sobre os aglomerados mas que não fazem parte dele dinamicamente) nas distribuições do espaço de fase (\textit{\textbf{shifting gapper}}) e estimado os raios,  dispersões de velocidades e massas ($R_{200}$, $\sigma$ e $M_{200}$, respectivamente, através do Teorema do Virial. Após a remoção de objetos duplos e dados espúrios, o catálogo final totaliza 7414 objetos.

A amostra final com essas propriedades e em baixo $z$ apresenta 127 agrupamentos. Posteriormente, a análise foi estendida para sistemas mais ricos, incluindo os aglomerados CIRS (\textit{Cluster Infall Regions in the SDSS}) \citeonline{rines2006cirs}, com 56 objetos. Portanto, a lista final  compreende 183 aglomerados (127 do NoSOCS e 56 do CIRS). 

\section{Catálogo III}
O modelo adotado segue o perfil de densidades $\rho(r)$ de um esferóide de Navarro-Frenk-White, conforme estudos por simulações numéricas cosmológicas realizadas por Navarro, Frenk e White \cite{NFW1997}. A partir dos resultados de simulações cosmológicas de N-corpos autogravitantes, sugere-se que o melhor ajuste universal do perfil radial de densidades que representam a distribuição de matéria de um aglomerado é dado por

\begin{equation}
\rho(r)=\rho_{crit}\frac{\delta_c}{(r/r_s)(1 + r/r_s)^2},
\label{nfw1}
\end{equation}

\noindent onde $\rho(r)$ é a densidade de matéria escura a uma distância $r$ do centro do esferóide, $\rho_{crit}$ é a densidade crítica de fundo do Universo no momento da formação do halo, $r_s$ é um raio característico da esfera, $\delta_c$ é uma sobredensidade característica do halo. Nas simulações descritas por \citeonline{NFW1997}, após a formação e maturação dos halos, verifica-se que estes estão em estado de equilíbrio dinâmico seguindo o modelo da dinâmica de N-Corpos autogravitantes em acordo com o teorema do Virial. Por outro lado, devido ao fato de que no modelo de \citeonline{NFW1997} $M(r) \to \infty$ para $r \to \infty$, os modelos de esferóide adotados neste trabalho são truncado em $R_{200}$, onde $R_{200}$ é a distância da região limítrofe da esfera na qual a densidade $\rho(r)$ é 200 vezes maior do que a densidade crítica do Universo $\rho_{crit}$. Também são geradas galáxias adicionais além de $R_{200}$ at\'e $2.5 R_{200}$, para simular objetos circunvizinhos a cada aglomerado de modo que ou estejam em processo de captura ou em estruturas filamentares observadas entre aglomerados.

Cada amostra é gerada através de um algoritmo composto por três laços : (I) O mais externo, (II) o primeiro interno, (III) o segundo interno. As unidades de medida usadas no código estão em $km \, s^{-1}$ para a velocidade, kpc para distância, para a constante gravitacional, G = 43007.1 ($kpc^3 \;10^{10}\; M_{\odot}^{-1}\; Gano^{-2}$), para constante de Hubble em $z=0$ é $H_0 = 0.069 \, km \, s^{-1} \, kpc^{-1}$.

Antes do primeiro laço, o código precisa como um dado de entrada um número inteiro como semente aleatória para iniciar o gerador de pseudo-números aleatórios usado ao longo de sua plena execução durante todo o código. 

O primeiro laço inicia-se com um dado solicitado como entrada outro número inteiro, agora correspondente ao número de sistemas da amostra simulada. Cada membro é um aglomerado pertencente à amostra. Em seguida, atribui-se um \textit{redshift} $z$ pseudo-aleatório no intervalo $0.03 \leq z \leq 0.13$ ao primeiro membro da amostra do aglomerado. Calcula-se depois o valor de $H(z)$, a constante de Hubble na época da virialização do aglomerado, com os parâmetros cosmológicos $\Omega_M = 0.3$ e $\Omega_{\Lambda}=0.7$, e a equação $H(z) = H_0*\sqrt{\Omega_M(1+z)^3 + \Omega_{\Lambda}}$.

A massa de cada aglomerado $M_{200}$ é atribuída no valor a partir de $10^{14} M_{\odot}$ até $10^{15.5}M_{\odot}$. O valor do número de membros (galáxias) do modelo $N_{200}$ dentro do raio virial $R_{200}$ é calculado conforme a seguinte equação obtida usando ajustes realizados em amostra de aglomerados \cite{andreon2012}
\begin{equation}
\log(N_{200}) = 0.47(\log(M_{200}) - 14.5) + 1.58,
\label{andreon}	
\end{equation}	
A partir do qual se calcula $N_{200} = e^{\log(N_{200})}$. Em seguida, calcula-se o valor da velocidade virial $v_{200}$\cite{springel1999}
\begin{equation}
v_{200} = \sqrt[3]{10GH(z)M_{200}} \, .
\label{velocidadevirial}
\end{equation}	
O valor de $R_{200}$ é dado por 
\begin{equation} R_{200} = \frac{v_{200}}{10H(z)}.
\label{rvirial}
\end{equation}
Estima-se o valor do parâmetro de concentração $c$ do esferóide, conforme uma prescrição obtida a partir de dados de aglomerados com massas $10^{11} \leq M_{200} \leq 10^{14} M_{\odot}$ \cite{Bullock}. \citeonline{Bullock} analisam dados de aglomerados simulados no modelo $\Lambda$CDM \cite{Bullock} a partir de simulações numéricas, enquanto Comerford e Natarajan analisam dados de aglomerados obtidos por observa\c c\~oes \cite{COMERFORD}. Os dados são compatíveis com o seguinte ajuste:
\begin{equation}
c = \frac{9.00}{(1+z)} \left(\frac{M_{200}}{1300}\right)^{(-0.13)} \, .	
\label{cparameter}
\end{equation}

Um dos dados usados para a construção dos aglomerados simulados é a escala característica, $r_c$. Ele é calculado pela equação que o define
\begin{equation}
r_c \equiv \frac{R_{200}}{c} \, .
\label{rc}
\end{equation}	

A dispersão de velocidades é dada por:
\begin{equation}
v_{disp} = \frac{GM_{200}}{R_{200}} \, .	
\label{vdisp}
\end{equation}
	
\noindent Em seguida, atribuem-se aleatoriamente os valores das coordenadas da ascenção reta e declinação do centróide do modelo, em condições de observação, como dados simulados. 


Após estes procedimentos, inicia-se o segundo laço (II) para atribuir pela técnica de Monte Carlo as posições e velocidades das galáxias de um aglomerado da amostra. 

As posições são determinadas resolvendo numericamente, pelo método da bissecção, a equação $q_{al} = \frac{M(r)}{M_{200}}$, onde $q_{al}$ é um número aleatório gerado pelo gerador de pseudo-números aleatórios. $M(r)$ é dado por
\begin{equation}
M(r)= 4 \pi \rho_{crit} \delta_c r_c^3 \left[ \frac{r_c + r}{r_c} - \frac{r}{r_c+r}  \right] \, .
\label{nfw2}
\end{equation}

Deste modo, $r$ é calculado  numericamente e, a partir dos ângulos gerados aleatoriamente em coordenadas esféricas $\theta$ e $\phi$, calculam-se as posições $x,y,z$ para a galáxia.

Para atribuir os dados da velocidade de modelos sem rotação, são geradas aleatoriamente em coordenadas esféricas $\theta$ e $\phi$. Em seguida, gerando uma distribuição gaussiana de velocidades para os componentes do vetor velocidade, para cada direção dos eixos coordenados-$xyz$ e, usando o valor da dispersão de velocidades $v_{disp}$, calculam-se $v_x$, $v_y$ e $v_z$. 

Para modelos com rotação, calcula-se a velocidade de rotação a partir da velocidade circular de cada galáxia a partir da equação 
\begin{equation}
v_{c} = \frac{G M(r)}{r},
\label{vcirc}
\end{equation}	
e o modelo é posto para rotacionar em torno do eixo-$z$.

Em cada modelo de aglomerado gerado, calcula-se a distância do centróide do objeto a um observador hipotético posicionado em $D(z)$ ao longo do eixo-$x$, dada pelo \textit{redshift z} pela equação: 

\begin{equation}
D(z) = \frac{c z}{H_0} \left(1-z\frac{(1+q_0)}{2}   \right), 	
\label{distanciaaglomerado}
\end{equation}

\noindent onde $c$ é a velocidade da luz no vácuo.
onde $q_0 = \Omega_{M}/2 - \Omega_{\Lambda}$ é o parâmetro de desaceleração usado no modelo cosmológico $\Lambda CDM$.

Cada modelo de aglomerado simulado possui $N_{200}$ galáxias. As posições cartesianas de cada galáxia são adicionadas ao centróide localizado na origem do sistema cartesiano. Em seguida, são projetadas as suas posições no plano-$yz$, interpretado como o planisfério celeste. As coordenadas cartesianas são convertidas em ascenção reta e declinação. Adicionalmente, para cada galáxia do aglomerado, calculam-se a projeção do vetor velocidade na linha de visada do observador e é convertido em \textit{redshift}, adicionado ao \textit{redshift} do aglomerado. 

Enfim, cálculos semelhantes são realizados para os objetos não pertencentes ao aglomerado. O segundo e terceiro laços se finalizam e o primeiro é finalizado após a geração de todos os aglomerados.


\section{Nosso Método}
O método foi baseado no trabalho de \citeonline{nascimento2016dynamical} que estudou o par de aglomerados A3407 + A3408, e que pode ser adaptado para o estudo da rotação de aglomerados individuais. 

Foi realizada uma análise na distribuição de velocidades das galáxias membro do aglomerado em busca de "gaps" significativos, como ilustrado da figura \ref{fig:selec20gap}. O gráfico na figura contém o histograma da distribuição de velocidades, o ajuste gaussiano superposto (linha em azul), barras inferiores indicando as velocidades individuais ordenadas em ordem crescente , sendo que em vermelho estão indicados os \textit{gaps} significativos.

\begin{figure}[H] %h or !htbp
\vspace{-2pt}
\begin{center}
\includegraphics[height=6.7cm,width=9cm]{04-figuras/selec20gap}%
\caption{Histograma Distribuição de Velocidade e Análise de Gaps para o aglomerado 08 do catálogo selec20.}
\label{fig:selec20gap}%
\end{center}
\end{figure}

O objetivo da rotina é identificar a probabilidade de que um \textit{"gap"},  de certo tamanho e em dada localização, possa ser produzido a partir de amostragens aleatórias retiradas de uma gaussiana. Para isto, as velocidades das galáxias são ordenadas em ordem crescente e o i-ésimo intervalo é definido como $g_i = v_{i+1} - v_i$. O "gap" é ponderado pela sua posição, através de $w_i=i(N-i)$, onde $N$ é o número de galáxias do aglomerado. Os "gaps" ponderados são então redefinidos por meio da média (MM) da distribuição ordenada do "gap" ponderado dada por: 

\begin{equation}
MM = \frac{2}{N} \sum_{i=N/4}^{3N/4} \sqrt{w_i g_i}
\label{eq:gappomderado}
\end{equation}

Investigamos "gaps" com valores maiores que 2.25, uma vez que em retiradas aleatórias de uma gaussiana, \textit{"gaps"} desse tamanho ocorrem no máximo em 3\% dos casos (vide \citeonline{wainer1978gapping} e \citeonline{beers1991dynamical}). Em seguida os dados foram divididos em duas amostras, contendo objetos com velocidades maiores e menores que o maior "gap" encontrado, referidos aqui como amostras I e II. Para o caso de aglomerados onde não foi encontrado \textit{"gap"} significativo ($>$ 2.25), utilizamos a mediana dos dados como divisor das velocidades do sistema.


A partir de \textit{"gaps"} identificados na distribuição de velocidades, levantou-se o seguinte questionamento: podem eles indicar um gradiente de velocidade em toda distribuição espacial de galáxias? Para isso, estimamos o eixo principal do aglomerado como o resultante do ajuste de uma elipse aos dados projetados no plano do céu, como ilustrado na figura \ref{elipse}. O ajuste foi feito usando-se o pacote \textbf{ellipse} do R.

\begin{figure}[H] %h or !htbp
\vspace{-2pt}
\begin{center}
\includegraphics[height=6.7cm,width=9cm]{04-figuras/elipse}%
\caption{Ajuste da elipse e eixo principal da distribuição projetada no plano do céu.}
\label{elipse}%
\end{center}
\end{figure}

Na Figura \ref{elipse}, vemos o ajuste da elipse na distribuição (X,Y) projetada no plano do céu, com os pontos acima ($+$) e abaixo ($-$) em relação à posição do \textit{gap} principal, indicados em azul e vermelho, respectivamente, e calculamos a distância de cada objeto ao eixo principal. Neste aglomerado analisando os objetos acima do eixo contabilizamos 13 com sinais positivos e 86 negativos, já os objetos abaixo do eixo foram 33 positivos e 93 negativos. A distribuição de galáxias em torno do \textit{gap} de velocidades pode ser utilizada como um indicador indireto da presença ou não de rotação. Estudamos o quanto diferem espacialmente as galáxias de acordo com a sua posição em relação ao eixo principal. Com isso as amostras I e II foram então comparadas em relação a sua distribuição de duas maneiras: independente do eixo principal e em cada lado do eixo. Os testes de comparação de duas amostras utilizados foram o teste de Cramer 2D e o de Hotelling, dos pacotes \textbf{ Cramer} e \textbf{ Hotelling} do R, respectivamente, tendo como a hipótese nula que os pontos $+$ e $-$ foram retirados da mesma população. 

O teste de Cramer para duas amostras pode ser usado para dados univariados e multivariados, como neste trabalho. Para o cálculo do valor crítico uma rotina de \textit{bootstrap} é utilizada e métodos de permutação são usados para obter o valor-p do teste. O teste de Hotelling multivariado compara médias em duas amostras. A rejeição ou não da hipótese nula é feita em todos os casos testados neste trabalho para um nível de 95\% de confiança. 

Dado que as distribuições espaciais das amostras  I e II sejam distintas com 95\% de confiança em relação aos testes acima citados, interpretamos o resultado como sendo uma indicação indireta de rotação nos aglomerados. Ou seja, interpretamos objetos com velocidades acima e abaixo do $gap$ apresentam também
diferentes distribuições espaciais.
Para os aglomerados onde isto acontece, traçamos um perfil de velocidade de rotação ao longo da distância ao centro do aglomerado. A velocidade de rotação foi calculada de maneira cumulativa contra o raio projetado das galáxias de acordo com

\begin{equation}
\omega= \Delta V/R
\label{eq:eq10}
\end{equation}

\noindent onde $\Delta V$ é a diferença de velocidade entre os pontos $+$ e $-$ internos a $R$, para dados radiais em ordem crescente.

\begin{figure}[H] %h or !htbp
\vspace{-2pt}
\begin{center}
\includegraphics[height=6.7cm,width=9cm]{04-figuras/rotacao}%
\caption{Perfil de velocidade de rotação do aglomerado 08 do catálogo selec20.}
\label{rotacao}%
\end{center}
\end{figure}

No gráfico, a linha tracejada na horizontal refere-se à velocidade de rotação do aglomerado, identificada através do raio (R200) do aglomerado (intersecção da curva de rotação e o valor de R200). Uma vez obtido o valor da velocidade de rotação do aglomerado utilizamos a fórmula dada por \citeonline{lee1969shape} que investigam a dependência da forma e do teorema do virial agindo no movimento orbital. Para que a velocidade angular afete significativamente o cálculo da massa do aglomerado, qualquer aceleração centrífuga precisaria ser comparável à aceleração gravitacional. Através da equação \ref{eq:maxRotacao} encontramos o valor limite de detecção de velocidade de rotação do aglomerado, representada na figura \ref{rotacao} pela linha tracejada na vertical.

\begin{equation}
\omega \cong \omega_{orb} = \frac{1}{r^2} \sqrt{G M_g \Delta (1 - e^2)}, 
\label{eq:maxRotacao}
\end{equation}

\noindent onde $\omega$ é a velocidade de rotação, G é constante gravitacional, R e M são raio e massa do aglomerado, respectivamente, e $e$ é dado por 

\begin{equation}
e = 1 - \frac{b}{a}, 
\label{eq:exce}
\end{equation}

\noindent sendo $a$ e $b$ as distâncias máximas horizontal e vertical ao centro da elipse, respectivamente.

Como observado o valor da rotação encontrado para o Aglomerado 08 do catálogo selec20 é $1.7577e^{-17}~rad/s$. Este é um valor aparentemente baixo, porém significa uma velocidade de rotação de aproximadamente $867 ~km \, s^{-1}$, maior que a dispersão de velocidades deste sistema,
$\sigma = 536~km \, s^{-1}$. O valor é próximo  de $1000 ~km \, s^{-1}$, que é o limite da detectabilidade de movimentos de gás em dados de raio-X, determinado observacionalmente por \citeonline{dupke2006direct} no aglomerado de Centaurus.

O método descrito acima é resumido no fluxograma ilustrado na figura \ref{fluxograma}.

\begin{figure}[H] %h or !htbp
\vspace{-2pt}
\begin{center}
\includegraphics[height=8cm,width=15cm]{04-figuras/fluxograma}%
\caption{Fluxograma do método implementado.}
\label{fluxograma}%
\end{center}
\end{figure}

\section{Método de Hwang \& Lee - adaptado}

Para efeito de comparação, implementamos também o 
método proposto por \citeonline{hwang2007searching}, usualmente aplicado para identificar a rotação em aglomerados. Eles utilizam a relação senoidal para calcular o eixo de rotação ($\Theta_o$) e a velocidade de rotação ($v_{rot}$), através de:

\begin{equation}
 vp(v_{rot}, \Theta) = v_{sys} + v_{rot} . sin(\Theta - \Theta_o) ,
 \label{eq:hwanglee1}
\end{equation}

\noindent onde $v_p$ é a velocidade radial de cada galáxia devida à rotação do aglomerado, $v_{sys}$ é a velocidade peculiar do aglomerado e $\Theta$ é o ângulo projetado na posição de cada galáxia no plano céu, partindo do Norte para o Leste. Usando-se diferenças de velocidade em relação à velocidade média do aglomerado, o valor de $v_{sys}$ foi definido como 0.

O procedimento de minimização do $\chi^2$ foi utilizado para determinar o melhor ajuste dos valores de $v_{rot}$ e $\Theta_0$, representado na equação \ref{hwanglee2}. Ou seja, o conjunto de valores de $v_{rot}$ e $\Theta_0$, são empregados no cálculo de $\chi^2$ para cada par de parâmetros:

\begin{equation}
 \chi^2 (v_{rot}, \Theta_o) = \sum_i{\frac{(v_{pi} - v_{los, i})^2}{\sigma^{2}_{i}}} ,
 \label{hwanglee2}
\end{equation}
onde $v_{los, i}$ é a velocidade de linha de visada de cada galáxia e $\sigma_i$ é a medida em erro.
Um aspecto importante do método HL é que o procedimento de minimização do $\chi^2$ sempre leva a um ajuste da curva, e 
portanto, em princípio todos os aglomerados possuem alguma rotação. O que distingue os aglomerados na verdade
é o grau de rotação. Os casos em que as velocidades de rotação sejam muito baixas para afetar a dinâmica dos aglomerados (ou aproximadamente zero)
são considerados exemplos de sistemas sem rotação. O fluxo resumido do método é representado na figura \ref{hlfluxo}.  

\begin{figure}[H] %h or !htbp
\vspace{-2pt}
\begin{center}
\includegraphics[height=8cm,width=8cm]{04-figuras/hlfluxo}%
\caption{Fluxograma do método Hwang \& Lee.}
\label{hlfluxo}%
\end{center}
\end{figure}

No próximo capítulo, apresentamos os resultados da aplicação dos métodos descritos acima.

\chapter{Análise}

Neste trabalho usamos 1000 réplicas dos dados de velocidade de cada catálogo com intuito de verificar a robustez do método na identificação ou não de rotação,
uuma vez que várias rotinas empregadas realizam aproximações e/ou fazem uso de sementes aleatórias. No catálogo I (selec20) 90\% dos casos (18 aglomerados) obtivemos resultados conclusivos, ou seja, em 900 delas encontramos o mesmo resultado, ou seja, a detecção ou não de rotação. Já no catálogo NoSocs, e nos Modelos III sem e com rotação obtivemos resultados conclusivos em 97.8\% (179 aglomerados), 97\% (194 aglomerados) e 100\% dos casos, respectivamente. Nas seções subsequentes apresentamos os resultados para cada catálogo.     

\section{Catálogo I: selec20}
\textbf{NOSSO MÉTODO}

Os resultados completos da aplicação de nosso método para o catálogo selec20 são apresentados no Anexo \ref{chap:anexoselec20} e na tabela \ref{tab:selec20T}. Na Figura \ref{selec20gap}, é exibida a análise de \textit{gaps} para um dos 20 aglomerados da amostra, o aglomerado 08. O gráfico na figura contém o histograma da distribuição de velocidades, o ajuste gaussiano superposto (linha em azul), barras inferiores indicando as velocidades individuais ordenadas em ordem crescente , sendo que em vermelho estão indicados os \textit{gaps} com valores maiores que 2.25, ou seja, os \textit{gaps} significativos. Quando mais de um \textit{gap} é encontrado, escolhemos aquele de maior valor; finalmente, a linha vertical tracejada indica a posição da BCG (\textit{brightest cluster galaxy}) apenas como referência. 

\begin{figure}[H] %h or !htbp
\vspace{-2pt}
\begin{center}
\includegraphics[height=6.7cm,width=9cm]{04-figuras/selec20gap}%
\caption{Histograma Distribuição de Velocidade e Análise de Gaps para o aglomerado 08.}
\label{selec20gap}%
\end{center}
\end{figure}

Na Figura \ref{elipse}, vemos o ajuste da elipse na distribuição (X,Y) projetada no plano do céu, com os pontos acima ($+$) e abaixo ($-$) em relação à posição do \textit{gap} principal, indicados em azul e vermelho, respectivamente. A distribuição de galáxias em torno do \textit{gap} de velocidades pode ser utilizada como um indicador indireto da presença ou não de rotação. Estudamos o quanto diferem espacialmente as galáxias de acordo com a sua posição em relação ao eixo principal. A hipótese nula dos testes é a de que os pontos $+$ e $-$ foram retirados da mesma população. Aplicamos dois testes estatísticos, Cramer 2D e Hotelling, em três cenários diferentes: todos os pontos do gráfico, acima e abaixo do eixo principal (Tabela \ref{tab:selec20T}). 

\begin{table}[H]
\caption{Teste Cramer e Hotelling aplicado no catálogo selec20.}
\vspace{12pt}
\centering{}
\resizebox{0.85\textwidth}{!}{
\begin{tabular}{lllllllll}
\hline
\multicolumn{1}{c}{\multirow{2}{*}{\textbf{Cluster}}}  		& 
\multicolumn{2}{|c}{\textbf{Cenário 1}}                     & 
\multicolumn{2}{c}{\textbf{Cenário 2}}                      & 
\multicolumn{2}{c|}{\textbf{Cenário 3}}                     & 
\multicolumn{1}{c|}{\multirow{2}{*}{\textbf{$ V_{rot} (km \, s^{-1})$}}}    & 
\multicolumn{1}{c}{\multirow{2}{*}{\textbf{Nº galáxias}}}  \\ \cline{2-7}
\multicolumn{1}{c}{}                                        & 
\multicolumn{1}{|c}{\textbf{Cramer}}                        &                 
\multicolumn{1}{c}{\textbf{Hotelling}}                      & 
\multicolumn{1}{c}{\textbf{Cramer}}                         & 
\textbf{Hotelling} & \textbf{Cramer}                        & 
\multicolumn{1}{l|}{\textbf{Hotelling}}                     & 
\multicolumn{1}{c|}{}                                       & 
\multicolumn{1}{c}{}                                    
\\ \hline
01  &  0.4965 & 0.6160 & 0.7682 &  0.8717 & 0.3646 & 0.7469 & - & 223 \\ 
02  &  {\color{red}0.0039} & {\color{red}0.0078} &  0.0769 &  {\color{red}0.0069} & {\color{red}0.0219} & 0.0916 & 1014.69 & 244 \\ 
03  &  0.2987 & 0.5612 &  0.6483 &  0.2024 & 0.0609 & 0.1299 & - & 177 \\   
04  &  {\color{red}<0.0001}  & {\color{red}<0.0001} & {\color{red}<0.0001} &  {\color{red}<0.0001} & {\color{red}0.0069} & {\color{red}0.01569} & 816.91 & 435 \\ 
05  &  0.0979 & {\color{red}0.0483} & {\color{red}0.0449} & 0.0723 & 0.0829 & 0.3963 & 1284.33 & 951\\ 
06  &  0.2817 & 0.2650 &  0.4725 &  0.8041 & 0.2577 & 0.4140 & - & 165 \\ 
07  &  {\color{red}0.0419} & 0.5880 & 0.5564 &  0.6346 & {\color{red}0.0389} & 0.8842 & 1878.89 & 313 \\ 
08  &  {\color{red}0.0099} & 0.4446 & {\color{red}0.0119} & 0.0807 & 0.0719 & 0.0551 & 824.97 & 225 \\ 
09  &  {\color{red}0.0289} & 0.2301 & {\color{red}0.0359} & {\color{red}0.0373} & 0.0959 & {\color{red}0.0364} & 837.38 & 215 \\ 
10  &  {\color{red}0.0009} & {\color{red}0.0067} &  {\color{red}0.0039} & {\color{red}0.0258} & 0.5024 & 0.8036 & 1232.28 & 308 \\ 
11  &  {\color{red}0.0009} & {\color{red}0.0031} &  0.3356 &  0.9006 & {\color{red}0.0009} & 0.0696 & 1154.06 & 651 \\ 
12  &  {\color{red}<0.0001}  & {\color{red}<0.0001} & {\color{red}<0.0001} & {\color{red}<0.0001} & {\color{red}<0.0001} &  {\color{red}<0.0001} & 2728.77 & 173 \\ 
13  &  0.1068 & 0.6982 &  0.1338 &  0.1516 & 0.2657 & 0.3324 & - & 218 \\ 
14  &  {\color{red}0.0389} & {\color{red}0.0054} & 0.4755 &  0.2110 & 0.7602 & 0.5019 & - & 127 \\ 
15  &  0.1178 & 0.5744 &  0.3416 &  0.2242 & {\color{red}0.0369} & 0.0609 & - & 114 \\ 
16  &  {\color{red}0.0449} & 0.5181 & {\color{red}0.0029} &  {\color{red}0.0194} & 0.4085 & 0.4775 & 781.13 & 144  \\ 
17  &  {\color{red}<0.0001}  & {\color{red}<0.0001} & {\color{red}<0.0001} & {\color{red}<0.0001} & 0.23676 & 0.3371 & 2239.97 & 773 \\ 
18  &  {\color{red}<0.0001} & {\color{red}0.0427} & {\color{red}0.0149} &  {\color{red}0.0075} & 0.1298 & 0.7534 & - & 233 \\ 
19  &  0.0879 & 0.4063 &  0.2997 &  0.2953 & 0.1378 & 0.2396 & - & 113 \\ 
20  &  0.1188 & 0.9037 &  0.0729 &  0.3173 & 0.1288 & 0.2297 & - & 210 \\ \hline
\label{tab:selec20T}
\end{tabular}
}
\end{table}

 O teste de Hotelling multivariado compara médias em duas amostras. A rejeição ou não da hipótese nula é feita em todos os casos para um nível de 95\% de confiança. Consideramos evidência significativa de rotação se houve rejeição da hipótese nula em pelo menos um dos cenários testadas. Isto nos leva a 14 aglomerados com evidência de algum grau de rotação. São eles os aglomerados: 02, 04, 05, 07, 08, 09, 10, 11, 12, 14, 15, 16, 17 e 18. Em uma análise mais criteriosa verificamos que:

\begin{itemize}
   	\item 14.29\% dos aglomerados houve rejeição da hipótese nula (indicando rotação) em pelo menos um dos testes (Cramer e Hotelling) nos três cenários (todos os pontos, acima e abaixo do eixo principal). 
   	\item 35.71\% dos aglomerados houve rejeição da hipótese nula em pelo menos um dos testes em dois cenários.
   	\item 7.14\% dos aglomerados houve rejeição da hipótese nula nos dois testes em apenas um cenário.
   	\item 7.14\% dos aglomerados houve rejeição da hipótese nula em pelo menos um dos testes em apenas um cenário.
   	\item 14.29\% dos aglomerados houve rejeição da hipótese nula em em ambos os testes nos três cenários.
   	\item 21.42\% dos aglomerados houve rejeição da hipótese nula em em ambos os testes em dois cenários.
 \end{itemize} 

\noindent Esta subdivisão permite que os usuários do programa analisem os resultados com maior ou menor conservadorismo.
Quando se deseja cometer menos falsos positivos na análise, devemos considerar que 42.86\% da amostra de aglomerados não têm rotação.
No extremo oposto, relaxando a interpretação, poderíamos concluir que apenas 7.14\% dos aglomerados não possuem rotação.

Finalmente, calculamos o perfil de velocidade de rotação para estes quatorze aglomerados, como ilustrado na Figura \ref{rotacao}. A velocidade de rotação foi calculada de maneira cumulativa contra o raio projetado das galáxias de acordo com

\begin{equation}
\omega= \Delta V/R
\label{eq:eq10}
\end{equation}

\noindent onde $\Delta V$ é a diferença de velocidade entre os pontos $+$ e $-$ internos a $R$.

No gráfico o valor exibido é a velocidade de rotação do aglomerado determinado a partir do R200 e a linha na vertical, o valor em y, é a velocidade angular de rotação no aglomerado, calculado pela fórmula da equação \ref{eq:maxRotacao}. 

\textbf{MÉTODO DE HWANG \& LEE}

Aplicamos o método de Hwang \& Lee para o catálogo selec20 e os resultados são apresentados na tabela \ref{tab:selec20hwang}. Para todos os aglomerados foi detectado um grau de rotação, dada a construção do método.

 \begin{table}[H]
\caption{Resultado do métdo Hwang \& Lee para o catálogo selec20.}
\vspace{12pt}
\centering{}
\resizebox{.8\textwidth}{!}{
\begin{tabular}{ccccc}
\cline{1-4}
\textbf{Cluster} & \textbf{Velocidade Rotacional} & \textbf{Ângulo (radiano)} & \textbf{Ângulo (grau)} \\ \hline
1 & -724.22 & 2.307 & 132.181 \\
2 & -661.24 & 2.715 & 155.558 \\
3 & -1016.65 & 1.767 & 101.242 \\
4 & 750.69 & 2.092 & 119.863 \\
5 & -790.08 & 1.197 & 68.583 \\
6 & 656.37 & 0.785 & 44.977 \\
7 & -1067.90 & 2.638 & 151.146 \\
8 & 852.35 & 0.14 & 8.021 \\
9 & 703.58 & 2.187 & 125.306 \\
10 & -873.78 & 1.678 & 96.142 \\
11 & -1117.95 & 2.803 & 160.6 \\
12 & -946.77 & 3.069 & 175.841 \\
13 & -891.43 & 2.461 & 141.005 \\
14 & -625.88 & 0.474 & 27.158 \\
15 & 864.45 & 0.278 & 15.928 \\
16 & -1041.45 & 1.011 & 57.926 \\
17 & -505.15 & 2.556 & 146.448 \\
18 & 889.73 & 0.812 & 46.524 \\
19 & -1062.70 & 3.029 & 173.549 \\
20 & 944.22 & 1.909 & 109.378 \\ \hline
\end{tabular}
}
\label{tab:selec20hwang}
\end{table}

Com intuito de comparar os valores das velocidades de rotação encontradas em cada um dos métodos para todos os aglomerados do catálogo selec20, construímos o gráfico ilustrado na figura \ref{vrotselec20}.

\begin{figure}[H] %h or !htbp
\vspace{-2pt}
\begin{center}
\includegraphics[scale=0.30]{04-figuras/selec20vrot}%
\caption{Comparação das velocidades de rotação encontradas em cada um dos métodos para catálogo selec20.}
\label{vrotselec20}%
\end{center}
\end{figure}

\section{Catálogo II: NoSOCS}
\textbf{NOSSO MÉTODO}

No catálogo de NoSOCS, diferente do selec20, 56.33\% dos aglomerados (um total de 89) não apresentaram \textit{gaps} significativos, nestes casos os dados foram divididos pelo cálculo da mediana. Além disso, 13.66\%, cerca de 25 aglomerados, continham um total inferior a 20 objetos, o que tornava o cálculo de detecção de rotação inviável. Por esse motivo, reduzimos a amostra de 183 para 158 aglomerados.

Aplicamos os mesmos cenários e testes usados no catálogo selec20 e verificamos que:

\begin{itemize}
   	\item 34.43\% dos aglomerados apresentaram indicação indireta de rotação, levando em conta um dos testes (Cramer e Hotelling) em pelo menos um cenário. Considerando-se que 47.62\% apenas as amostras sem \textit{gap}, ou seja, em casos onde os dados foram divididos pela mediana. 
   	\item Desse total, tendo em conta a rejeição da hipótese em ambos os testes para os três cenários um total de 6.35\%, dois cenários 15.87\% e um cenário 9.52\%.
   	\item Já a rejeição da hipótese em pelo menos um dos testes para os três cenários um total de 11.11\%, dois cenários 15.87\% e um cenário 41.27\%. 
 \end{itemize} 

 Os resultados são apresentados nas tabelas \ref{tab:nosocsI} e \ref{tab:nosocsII} e no Anexo \ref{chap:anexonosocs}. 

\begin{table}[H]
\caption{Teste Cramer e Hotelling aplicado no catálogo NoSOCS com rotação detectada utilizando gap.}
\vspace{12pt}
\centering{}
\resizebox{.9\textwidth}{!}{
\begin{tabular}{lllllllll}
\hline
\multicolumn{1}{c}{\multirow{2}{*}{\textbf{Cluster}}}  		& 
\multicolumn{2}{|c}{\textbf{Cenário 1}}                     & 
\multicolumn{2}{c}{\textbf{Cenário 2}}                      & 
\multicolumn{2}{c|}{\textbf{Cenário 3}}                     & 
\multicolumn{1}{c|}{\multirow{2}{*}{\textbf{$ V_{rot} (km \, s^{-1})$}}}    & 
\multicolumn{1}{c}{\multirow{2}{*}{\textbf{Nº galáxias}}}  \\ \cline{2-7}
\multicolumn{1}{c}{}                                        & 
\multicolumn{1}{|c}{\textbf{Cramer}}                        &                 
\multicolumn{1}{c}{\textbf{Hotelling}}                      & 
\multicolumn{1}{c}{\textbf{Cramer}}                         & 
\textbf{Hotelling} & \textbf{Cramer}                        & 
\multicolumn{1}{l|}{\textbf{Hotelling}}                     & 
\multicolumn{1}{c|}{}                                       & 
\multicolumn{1}{c}{}                                    
 \hline
01238 & {\color{red}0.0039}	& {\color{red}0.0056} & {\color{red}0.0219} & {\color{red}0.0252} & 0.0569 & 0.1505 & 797.69 & 44 \\ 
01836 & {\color{red} < 0.0001} & {\color{red}0.0003} & {\color{red}0.0069} & {\color{red}0.0108} &  {\color{red}0.0229} & {\color{red}0.0012} & 448.87 & 86 \\ 
02440 & {\color{red}0.0179} & {\color{red}0.0906} & {\color{red}0.0699} & {\color{red}0.0057} & {\color{red}0.1508} & {\color{red}0.1378} & 302.64 & 41 \\ 
02447 & 0.5114 & 0.6465 & {\color{red}0.0319} & 0.0532 & 0.7432 & 0.5480 & - & 46 \\ 
04681 & {\color{red}0.0149} & {\color{red}0.0174} & {\color{red}0.0249} & {\color{red}0.0478} & 0.2047 & 0.2128 & 1084.22 & 162 \\ 
04703 & {\color{red}0.0019} & {\color{red}0.0005} & {\color{red}0.0279} & {\color{red}0.0186} & 0.1098 & 0.1499 & 1147.11 & 91 \\ 
06256 & {\color{red}0.0359} & {\color{red}0.0166} & {\color{red}0.0269} & {\color{red}0.0133} & 0.2837 & 0.3682 & 665.03 & 37 \\ 
06547 & {\color{red} < 0.0001} & {\color{red}0.0052} & 0.5624 & 0.8490 & {\color{red}0.0019} & {\color{red}0.0050} & 1104.74 & 112 \\ 
07217 & 0.0779 & {\color{red}0.0384} & 0.3826 & 0.4338 & 0.1528 & 0.4565 & - & 24 \\ 
07703 & {\color{red}0.0439} & 0.0796 & {\color{red}0.0059} & {\color{red}0.0151} & 0.5864 & 0.9808 & 308.04 & 48 \\ 
09148 & {\color{red}0.0129} & {\color{red}0.0044} & {\color{red}0.0489} & 0.1124 & {\color{red}0.0229} & 0.0891 & - & 31 \\ 
10008 & {\color{red}0.0019} & 0.2934 & {\color{red}0.0299} & 0.3870 & {\color{red}0.0259} & 0.1435 & 1524.44 & 123 \\ 
10013 & {\color{red}0.0089} & {\color{red}0.0313} & 0.2877 & 0.8329 & 0.3776 & 0.5823 & 933.65 & 86 \\ 
10015 & {\color{red}0.0039} & 0.0576 & 0.0929 & 0.2795 & {\color{red}0.0299} & 0.0574 & 1291.87 & 139 \\ 
10020 & {\color{red}0.0009} & {\color{red}0.0086} & {\color{red}0.0039} & 0.4036 & {\color{red}0.0049} & {\color{red}0.0166} & 1195.4 & 114 \\ 
10021 & {\color{red}0.0049} & {\color{red}0.0305} & {\color{red}0.0129} & {\color{red}0.0152} & NA & NA & - & 89 \\ 
10023 & {\color{red}0.0409} & 0.0848 & 0.0789 & 0.2775 & 0.1108 & 0.2771 & 1738.54 & 107 \\ 
10024 & {\color{red}0.0379} & 0.0749 & 0.0909 & 0.2553 & 0.2317 & 0.4133 & 1873.25 & 113 \\ 
10027 & 0.1638 & 0.4139 & 0.2687 & 0.3932 & {\color{red}0.0359} & 0.0593 & 824.85 & 86 \\ 
10029 & {\color{red}0.0259} & 0.0564 & {\color{red}0.0489} & 0.1656 & 0.0579 & 0.0581 & 505.62 & 135 \\ 
10030 & {\color{red} < 0.0001} & > 1.0 & {\color{red}0.0139} & {\color{red}0.0152} & {\color{red} < 0.0001} & > 1.0 & 323.01 & 87 \\ 
10037 & {\color{red}0.0009} & {\color{red}0.0089} & {\color{red}0.0329} & 0.2859 & 0.0709 & 0.1575 & 1141.23 & 145 \\ 
10043 & {\color{red}0.0209} & 0.0599 & 0.0539 & 0.0616 & 0.1888 & 0.2711 & 686.59 & 218 \\ 
10044 & {\color{red} < 0.0001} & > 1.0 & {\color{red}0.0039} & {\color{red}0.0081} & {\color{red}0.0019} & {\color{red}0.0010} & 1431.5 & 177 \\ 
10045 & 0.1038 & 0.2814 & {\color{red}0.0379} & 0.0935 & 0.4815 & 0.9611 & 450.57 & 55 \\ 
10051 & {\color{red}0.0459} & 0.2409 & {\color{red} < 0.0001} & {\color{red}0.0037} & 0.2147 & 0.0671 & 857.33 & 189 \\ 
10053 & 0.0659 & 0.5593 & {\color{red} < 0.0001} & {\color{red}0.0001} & {\color{red}0.0019} & {\color{red}0.0010} & 981.84 & 447 \\ 
10054 & {\color{red} < 0.0001} & {\color{red}0.0068} & 0.2017 & 0.5760 & {\color{red} < 0.0001} & {\color{red}0.0107} & 1230.39 & 537 \\ 
10056 & {\color{red}0.0259} & {\color{red}0.0209} & 0.1608 & {\color{red}0.0471} & 0.3186 & 0.3730 & 1457.95 & 100 \\ 
10059 & {\color{red}0.0419} & 0.4240 & 0.1118 & 0.2904 & 0.3096 & 0.4049 & - & 150 \\ 
10060 & 0.2017 & 0.9175 & {\color{red}0.0059} & {\color{red}0.0485} & 0.8311 & 0.6474 & 803.62 & 133 \\ 
10063 & 0.0999 & 0.6957 & {\color{red}0.0499} & 0.1891 & {\color{red}0.0339} & 0.1027 & 1236.38 & 219 \\ 
10064 & 0.0969 & {\color{red}0.0428} & 0.2797 & 0.1394 & 0.6283 & 0.7897 & 1122.04 & 133 \\ \hline
\label{tab:nosocsI}
\end{tabular}
}
\end{table}

\begin{table}[H]
\caption{Teste Cramer e Hotelling aplicado no catálogo NoSOCS com rotação detectada utilizando mediana.}
\vspace{12pt}
\centering{}
\resizebox{.9\textwidth}{!}{
\begin{tabular}{lllllllll}
\hline
\multicolumn{1}{c}{\multirow{2}{*}{\textbf{Cluster}}}  		& 
\multicolumn{2}{|c}{\textbf{Cenário 1}}                     & 
\multicolumn{2}{c}{\textbf{Cenário 2}}                      & 
\multicolumn{2}{c|}{\textbf{Cenário 3}}                     & 
\multicolumn{1}{c|}{\multirow{2}{*}{\textbf{$ V_{rot} (km \, s^{-1})$}}}    & 
\multicolumn{1}{c}{\multirow{2}{*}{\textbf{Nº galáxias}}}  \\ \cline{2-7}
\multicolumn{1}{c}{}                                        & 
\multicolumn{1}{|c}{\textbf{Cramer}}                        &                 
\multicolumn{1}{c}{\textbf{Hotelling}}                      & 
\multicolumn{1}{c}{\textbf{Cramer}}                         & 
\textbf{Hotelling} & \textbf{Cramer}                        & 
\multicolumn{1}{l|}{\textbf{Hotelling}}                     & 
\multicolumn{1}{c|}{}                                       & 
\multicolumn{1}{c}{}                                        \\ \hline
01831 & {\color{red} < 0.0001} & {\color{red}0.0026} & {\color{red}0.0019} & {\color{red}0.0312} & {\color{red}0.0099} & {\color{red}0.0040} & 345.55 & 72 \\
02137 & {\color{red}0.0369} & 0.3470 & 0.2487 & 0.3295 & 0.1618 & 0.7326 & - & 34 \\
02433 & 0.1188 & 0.3012 & 0.8461 & NA & {\color{red}0.0329} & 0.2165 & 593.83 & 31 \\
03176 & {\color{red}0.0409} & 0.4379 & 0.1558 & 0.1618 & 0.1118 & 0.7231 & 428.81 & 26 \\
03691 & 0.0609 & {\color{red}0.0318} & 0.4645 & 0.5344 & 0.4355 & 0.1154 & - & 43 \\
03907 & 0.1148 & 0.1686 & {\color{red}0.0159} & 0.0652 & 0.6063 & 0.9985 & 547.83 & 44 \\
04048 & 0.1398 & {\color{red}0.0232} & 0.2527 & 0.2135 & 0.8171 & 0.5988 & - & 30 \\
04409 & {\color{red}0.0009} & {\color{red}0.0012} & 0.1018 & 0.1781 & {\color{red}0.0069} & {\color{red}0.01041} & - & 23 \\
04470 & {\color{red}0.0479} & 0.1103 & 0.5114 & NA & 0.0569 & 0.1361 & 775.23 & 35 \\
04479 & 0.0909 & 0.1737 & 0.8301 & 0.0699 & {\color{red}0.0139} & {\color{red}0.0205} & - & 25 \\
04672 & 0.3546 & 0.6005 & 0.5404 & 0.4337 & 0.1688 & {\color{red}0.0242} & - & 48 \\
05359 & 0.0969 & 0.1166 & {\color{red}0.0109} & {\color{red}0.0250} & 0.4835 & 0.7024 & 656.00 & 62 \\
05447 & {\color{red} < 0.0001} & {\color{red}0.0010} & {\color{red}0.0239} & {\color{red}0.0463} & {\color{red}0.0079} & {\color{red}0.0156} & 497.66 & 57 \\
06207 & 0.6263 & 0.3789 & 0.2682 & {\color{red}0.0077} & 0.7092 & 0.5950 & - & 21 \\
06286 & 0.1298 & 0.0540 & {\color{red}0.0199} & {\color{red}0.0133} & 0.1308 & 0.1592 & 673.53 & 70 \\
07395 & 0.3286 & 0.7779 & 0.4095 & 0.6738 & 0.1368 & {\color{red}0.0007} & - & 29 \\
08022 & 0.2667 & 0.8572 & {\color{red}0.0409} & 0.2593 & 0.1478 & 0.1502 & 650.79 & 51 \\
08721 & {\color{red}0.0329} & 0.1291 & 0.2257 & 0.5158 & 0.0849 & 0.2758 & 390.67 & 47 \\
09061 & 0.3026 & {\color{red}0.0371} & 0.8051 & 0.8967 & 0.5174 & 0.0671 & - & 46 \\
10006 & {\color{red}0.0409} & 0.0847 & 0.1958 & 0.6024 & {\color{red}0.0189} & {\color{red}0.0057} & 932.21 & 74 \\
10016 & {\color{red}0.0009} & {\color{red}0.0008} & {\color{red}0.0349} & 0.0727 & {\color{red}0.0009} & {\color{red}0.0056} & - & 92 \\
10018 & 0.3986 & 0.3313 & {\color{red}0.0479} & 0.8072 & 0.3306 & 0.1415 & 748.49 & 65 \\
10026 & {\color{red}0.0339} & {\color{red}0.0363} & 0.5314 & 0.6806 & {\color{red} < 0.0001} & {\color{red}0.0018} & - & 130 \\
10028 & {\color{red}0.0189} & {\color{red}0.0187} & 0.2317 & 0.2742 & 0.0539 & 0.0525 & 830.93 & 80 \\
10031 & {\color{red} < 0.0001} & {\color{red}0.0006} & {\color{red}0.0029} & {\color{red}0.0010} & {\color{red} < 0.0001} & {\color{red}0.0001} & 1070.37 & 141 \\
10035 & 0.1228 & 0.1567 & 0.5054 & 0.9967 & {\color{red}0.0299} & 0.0745 & 807.11 & 58 \\
10036 & 0.4105 & 0.9175 & 0.0569 & {\color{red}0.0455} & 0.6703 & 0.5445 & 647.85 & 119 \\
10040 & {\color{red}0.0009} & {\color{red}0.0019} & {\color{red}0.0369} & 0.0593 & {\color{red}0.0079} & {\color{red}0.0099} & 635.65 & 99 \\
10050 & {\color{red}0.0209} & 0.5951 & 0.0689 & 0.2795 & 0.0779 & 0.0509 & 795.6 & 77 \\
10055 & {\color{red} < 0.0001} & 0.3621 & 0.2607 & 0.1808 & {\color{red} < 0.0001} & {\color{red}0.0007} & - & 104 \\ \hline

\label{tab:nosocsII}
\end{tabular}
}
\end{table}

\textbf{MÉTODO DE HWANG \& LEE}

Aplicamos o método de Hwang \& Lee para o catálogo NoSocs e os resultados são apresentados na tabela \ref{tab:nosocshwang}. Cerca de 97.26\% dos aglomerados (178) detectaram rotação. Apenas nos aglomerados 02907, 08049, 08720, 09144 e 10046 não identificaram grau de rotação.

{\scriptsize
\begin{longtable}{ccccc}
\caption{Resultado do métdo Hwang \& Lee para o catálogo NoSocs.}\label{tab:nosocshwang}
\\ \hline
\textbf{Cluster} & \textbf{Velocidade Rotacional} & \textbf{Ângulo (radiano)} & \textbf{Ângulo (grau)} \\ \hline
00053 & -587.01 & 3.141 & 179.981\\
00086 & 53.71 & 0 & 0\\
00339 & -1184.30 & 3.044 & 174.443\\
00719 & 488.11 & 0.826 & 47.363\\
00996 & 969.00 & 0.966 & 55.379\\
01052 & 328.83 & 1.808 & 103.625\\
01189 & -660.52 & 2.738 & 156.907\\
01238 & -550.89 & 0.146 & 8.371\\
01264 & 782.77 & 0.628 & 35.996\\
01347 & 644.06 & 3.141 & 179.981\\
01478 & 801.68 & 0.589 & 33.746\\
01831 & -1184.30 & 1.681 & 96.328\\
01836 & -356.52 & 2.365 & 135.515\\
01877 & 665.08 & 1.861 & 106.655\\
01933 & 643.25 & 2.203 & 126.221\\
02035 & -383.22 & 1.478 & 84.697\\
02104 & 940.52 & 2.339 & 134.029\\
02137 & 1016.62 & 1.713 & 98.171\\
02182 & 581.02 & 0.872 & 49.994\\
02186 & 631.46 & 2.513 & 143.985\\
02249 & 1085.40 & 1.256 & 71.992\\
02298 & 761.15 & 0.336 & 19.283\\
02301 & -816.24 & 2.250 & 128.905\\
02433 & -352.07 & 1.256 & 71.992\\
02440 & -503.39 & 0.392 & 22.497\\
02447 & 631.46 & 0.907 & 51.994\\
02469 & -1002.72 & 0.628 & 35.996\\
02490 & 789.35 & 1.912 & 109.554\\
02752 & 177.52 & 2.261 & 129.586\\
02759 & 672.72 & 1.142 & 65.447\\
02789 & 376.11 & 0.196 & 11.248\\
02827 & -116.20 & 0.369 & 21.174\\
02899 & -286.97 & 2.191 & 125.568\\
02907 &       - &     - & - \\
03031 & -771.62 & 3.141 & 179.981\\
03112 & 561.62 & 0.966 & 55.379\\
03176 & 722.24 & 0.628 & 35.996 \\
03229 & 436.91 & 0.538 & 30.854 \\
03459 & -1042.44 & 2.061 & 118.113\\
03565 & -404.09 & 0.490 & 28.122\\
03631 & -1184.30 & 2.792 & 159.983\\
03691 & -751.97 & 0.299 & 17.141\\
03742 & -842.29 & 0.387 & 22.189\\
03898 & -1184.30 & 0.649 & 37.237\\
03907 & -603.67 & 0.219 & 12.556\\
03915 & -324.56 & 1.047 & 59.993\\
03975 & -191.30 & 0.883 & 50.619\\
04023 & -715.55 & 2.902 & 166.287\\
04048 & 1007.13 & 1.733 & 99.300\\
04100 & 517.97 & 1.658 & 94.990\\
04376 & 761.15 & 1.907 & 109.274\\
04404 & 672.72 & 2.855 & 163.619\\
04405 & 217.57 & 0.092 & 5.29358\\
04409 & -977.96 & 2.141 & 122.714\\
04458 & 594.65 & 1.1038 & 63.236\\
04470 & -316.47 & 2.402 & 137.633\\
04479 & -238.59 & 1.963 & 112.488\\
04537 & -276.42 & 0 & 0\\
04641 & 328.83 & 1.256 & 71.992\\
04672 & -942.84 & 3.007 & 172.323\\
04681 & 958.52 & 0.468 & 26.829\\
04703 & 1034.96 & 1.780 & 101.989\\
04710 & 599.03 & 2.804 & 160.698\\
05039 & -860.05 & 0.448 & 25.711\\
05206 & 177.52 & 1.633 & 93.590\\
05325 & -690.88 & 2.868 & 164.331\\
05359 & -700.59 & 2.008 & 115.070\\
05447 & 315.32 & 0.504 & 28.925\\
05535 & -522.30 & 1.308 & 74.992\\
05717 & 916.01 & 1.828 & 104.765\\
05780 & 1085.40 & 0 & 0\\
05859 & 768.69 & 0.949 & 54.413\\
05908 & -745.00 & 0.405 & 23.223\\
06070 & 653.07 & 0.673 & 38.567\\
06173 & -1184.30 & 1.287 & 73.763\\
06175 & -276.42 & 2.827 & 161.983\\
06184 & -1016.17 & 2.210 & 126.653\\
06207 & -616.87 & 3.141 & 179.981\\
06233 & 697.89 & 0.996 & 57.067\\
06256 & -679.92 & 0.698 & 39.995\\
06261 & -1184.30 & 1.671 & 95.735\\
06264 & -197.47 & 1.775 & 101.728\\
06286 & -625.09 & 3.050 & 174.764\\
06392 & -377.29 & 1.186 & 67.993\\
06447 & 457.61 & 1.804 & 103.393\\
06475 & -739.26 & 1.108 & 63.523\\
06506 & 1039.08 & 1.218 & 69.788\\
06508 & -639.57 & 2.010 & 115.188\\
06547 & 942.26 & 1.698 & 97.287\\
06723 & -427.73 & 1.495 & 85.705\\
06758 & -697.93 & 2.917 & 167.125\\
06841 & 678.01 & 2.980 & 170.752\\
06861 & -616.87 & 0.392 & 22.497\\
06924 & -922.41 & 1.329 & 76.146\\
07204 & -771.62 & 0.999 & 57.266\\
07217 & 98.57 & 2.458 & 140.855\\
07395 & 274.79 & 1.907 & 109.274\\
07435 & -907.50 & 0.383 & 21.949\\
07520 & -427.73 & 3.141 & 179.981\\
07584 & 1085.40 & 2.792 & 159.983\\
07703 & -797.96 & 2.874 & 164.664\\
07775 & 917.27 & 2.908 & 166.649\\
07815 & 561.62 & 1.933 & 110.758\\
07837 & 653.07 & 2.543 & 145.699\\
07975 & -1119.45 & 0.448 & 25.711\\
08022 & 722.24 & 1.633 & 93.590\\
08049 & - & - & - \\
08173 & 1085.40 & 2.731 & 156.505\\
08219 & 722.24 & 0.879 & 50.394\\
08291 & -914.09 & 0.747 & 42.852\\
08338 & 684.86 & 3.141 & 179.981\\
08710 & -1039.42 & 1.069 & 61.270\\
08720 & - & - & - \\
08721 & -1184.30 & 2.526 & 144.768\\
08738 & -911.93 & 2.953 & 169.182\\
08742 & -896.08 & 2.692 & 154.270\\
08975 & 740.01 & 0.478 & 27.388\\
09061 & -1133.86 & 1.605 & 91.990\\
09075 & -706.46 & 2.976 & 170.509\\
09132 & -394.83 & 3.141 & 179.981\\
09144 & - & - & - \\
09148 & -427.73 & 2.094 & 119.987\\
09153 & 789.35 & 0.956 & 54.777\\
09157 & 490.95 & 3.066 & 175.696\\
09162 & 431.41 & 1.863 & 106.768\\
09176 & 730.75 & 1.996 & 114.363\\
09177 & 156.88 & 2.855 & 163.619\\
10001 & 362.21 & 0.554 & 31.761\\
10004 & -863.87 & 0.702 & 40.231\\
10006 & 525.74 & 1.506 & 86.292\\
10008 & -1128.48 & 0.283 & 16.227\\
10009 & -849.42 & 2.214 & 126.872\\
10010 & -659.22 & 2.321 & 132.971\\
10013 & -650.25 & 2.217 & 127.046\\
10014 & 1085.40 & 2.645 & 151.563\\
10015 & 509.75 & 0.614 & 35.213\\
10016 & 786.09 & 0.138 & 7.911\\
10017 & -1053.85 & 2.058 & 117.919\\
10018 & -1077.90 & 1.178 & 67.493\\
10019 & -1071.60 & 0.445 & 25.529\\
10020 & -441.12 & 0.500 & 28.669\\
10021 & 1085.40 & 1.213 & 69.538\\
10022 & 780.92 & 1.111 & 63.652\\
10023 & -841.70 & 1.718 & 98.480\\
10024 & -1042.44 & 2.047 & 117.309\\
10025 & -1020.88 & 3.066 & 175.662\\
10026 & 839.07 & 3.0685 & 175.796\\
10027 & 978.59 & 0.036 & 2.117\\
10028 & -695.88 & 0.119 & 6.834\\
10029 & 1034.58 & 3.071 & 175.952\\
10030 & -392.54 & 1.497 & 85.805\\
10031 & 890.85 & 1.121 & 64.279\\
10032 & -271.48 & 1.434 & 82.165\\
10033 & 537.54 & 0.361 & 20.687\\
10034 & 601.69 & 0.772 & 44.257\\
10035 & 926.12 & 0.936 & 53.678\\
10036 & -1165.06 & 2.955 & 169.304\\
10037 & -679.92 & 3.054 & 174.982\\
10038 & 615.34 & 0.111 & 6.389\\
10039 & 626.05 & 1.047 & 59.993\\
10040 & 575.87 & 0.160 & 9.182\\
10041 & -1042.44 & 3.141 & 179.981\\
10042 & -305.70 & 1.824 & 104.505\\
10043 & 196.34 & 2.576 & 147.634\\
10044 & 995.12 & 1.053 & 60.334\\
10045 & 1043.36 & 1.512 & 86.657\\
10046 & - & - & - \\
10047 & -782.58 & 1.612 & 92.380\\
10048 & -1158.98 & 0.992 & 56.871\\
10049 & -383.22 & 1.154 & 66.169\\
10050 & -587.01 & 1.240 & 71.045\\
10051 & 409.31 & 2.573 & 147.431\\
10052 & -276.42 & 0.773 & 44.303\\
10053 & 1003.97 & 1.387 & 79.498\\
10054 & -629.57 & 2.584 & 148.082\\
10055 & -875.79 & 0.335 & 19.221\\
10056 & -1092.59 & 1.872 & 107.261\\
10058 & -902.05 & 0.016 & 0.932\\
10059 & -194.16 & 2.466 & 141.328\\
10060 & 793.09 & 2.213 & 126.805\\
10062 & -906.37 & 1.666 & 95.500\\
10063 & 398.24 & 2.810 & 160.992\\
10064 & 414.80 & 1.285 & 73.628\\ \hline
\end{longtable}
}

Os valores das velocidades de rotação encontradas em cada um dos métodos pode ser vista no gráfico ilustrado na figura \ref{vrotnosocs}.

\begin{figure}[H] %h or !htbp
\vspace{-2pt}
\centering
\subfloat{\includegraphics[scale=.30]{04-figuras/nosocsivrot}}
\subfloat{\includegraphics[scale=.30]{04-figuras/nosocsiivrot}}
\caption{Comparação das velocidades de rotação encontradas em cada um dos métodos para catálogo NoSocs.}
\label{vrotnosocs}
\end{figure}

\section{Catálogo III}
\textbf{NOSSO MÉTODO}

Utilizamos o método descrito na seção \ref{subsec:simulate} na geração de duas amostras, com e sem rotação, que chamaremos de amostra I e II, respectivamente, cada uma com 200 elementos. Na amostra com rotação 59.5\% dos casos (119 aglomerados) não apresentaram \textit{gaps} significativos, portanto utilizamos a mediana. Já na amostra sem rotação o número de casos que não apresentaram \textit{gaps} significativos foi menor, 18.5\%, ou seja, 37 aglomerados.

Aplicamos os mesmos cenários e testes usados nos catálogos anteriores e identificamos que:

Para a amostra I:
\begin{itemize}
	\item 100\% dos aglomerados apresentaram indicação de rotação.
   	\item Desse total, a rejeição da hipótese em ambos os testes para um cenário foi um total de 6\%, dois cenários 5\% e nos três cenários 15\%.
   	\item Já a rejeição da hipótese em pelo menos um dos testes para um cenário foi um total de 4\%, dois cenários 11\% e os três cenários 59\%.
 \end{itemize} 

Para a amostra II:
\begin{itemize}
   	\item 13.5\% dos aglomerados apresentaram indicação indireta de rotação.
   	\item Desse total, a rejeição da hipótese em ambos os testes para um cenário foi um total aproximado de 37.04\%, dois cenários 3.7\%.
   	\item Já a rejeição da hipótese em pelo menos um dos testes para um cenário foi um total aproximado de 37.04\%, dois cenários 22.22\%. 
 \end{itemize} 

Devemos notar que os resultados para a amostra I permitem que entendamos melhor como analisar a saída do programa.
Como a amostra é controlada e todos os modelos simulados possuem rotação, nosso primeiro resultado é totalmente
consistente com isto, ou seja, de fato encontramos 100\% dos aglomerados com indicação de rotação. Se, no entanto, 
aplicamos os testes estatísticos sobre a distribuição espacial das galáxias, este número pode diminuir, dependendo 
da escolha. Isto sugere que o conservadorismo pode levar a algum nível de falsos negativos nos diagnósticos da amostra.
Por exemplo, se usarmos o critério de  rejeição da hipótese em ambos os testes para os três cenários investigados (15\%),
encontraríamos 30 aglomerados sem rotação, quando na verdade eles rotacionam. Esta consideração sugere 
menos conservadorismo em relação a falsos negativos.

Por outro lado, os resultados para a amostra II, com modelos sem rotação, sugerem que podemos ter considerável
fração de falsos positivos em nossos diagnósticos. Inicialmente, temos uma indicação de 13.5\%, ou seja, 27 aglomerados
sem rotação teriam indicação de rotação. Vamos adotar este número como um fator de erro (ou confiança) em nossos diagnósticos,
no que se refere a falsos positivos.


Os resultados para este catálogo são apresentados nas tabelas \ref{tab:sorirotation1} a \ref{tab:sorinorotation2} e no Anexo.

{\scriptsize
\begin{longtable}{cccccccc}
\hline
\multirow{2}{*}{\textbf{Cluster}} & \multicolumn{2}{|c|}{\textbf{Cenário 1}} & \multicolumn{2}{c|}{\textbf{Cenário 2}} & \multicolumn{2}{c|}{\textbf{Cenário 3}} & \multirow{2}{*}{\textbf{Nº galáxias}} \\ \cline{2-7}
                         & \multicolumn{1}{|c}{\textbf{Cramer}}       & \textbf{Hotelling}       & \textbf{Cramer}       & \textbf{Hotelling}       & \textbf{Cramer}       & \textbf{Hotelling}       &                              \\ \hline
002 & {\color{red}0.0119} & > 1.0 & - & - & 0.0919 & {\color{red}0.0024} & 22 \\
005 & {\color{red} < 0.0001} & {\color{red}0.0001} & 0.0999 & 0.1513 & {\color{red}0.0039} & {\color{red}0.0053} & 23 \\
008 & {\color{red}0.0029} & {\color{red}0.0001} & 0.2567 & 0.1049 & - & - & 23 \\
018 & {\color{red} < 0.0001} & > 1.0 & {\color{red} < 0.0001} & {\color{red}0.0002} & {\color{red}0.0119} & {\color{red}0.0119} & 25 \\
022 & {\color{red} < 0.0001} & > 1.0 & {\color{red}0.0389} & 0.0953 & {\color{red}0.0009} & {\color{red}0.0044} & 26 \\
030 & {\color{red} < 0.0001} & 0.9050 & {\color{red}0.0039} & {\color{red}0.0012} & {\color{red}0.0019} & {\color{red}0.0008} & 28 \\
031 & {\color{red} < 0.0001} & > 1.0 & {\color{red}0.0119} & NA & {\color{red} < 0.0001} & {\color{red}0.0010} & 28 \\
038 & {\color{red} < 0.0001} & 0.2570 & {\color{red}0.0029} & {\color{red}0.0023} & {\color{red}0.0029} & {\color{red}0.0079} & 30 \\
039 & {\color{red} < 0.0001} & > 1.0 & {\color{red}0.0099} & {\color{red}0.0088} & {\color{red}0.0019} & {\color{red}0.0002} & 30 \\
043 & {\color{red} < 0.0001} & 0.3010 & {\color{red}0.0039} & {\color{red}0.0057} & {\color{red}0} & > 1.0 & 31 \\
048 & {\color{red} < 0.0001} & > 1.0 & {\color{red}0.0019} & {\color{red}0.0061} & {\color{red} < 0.0001} & {\color{red}0.0061} & 32 \\
050 & {\color{red} < 0.0001} & > 1.0 & {\color{red}0.0029} & {\color{red}0.0007} & {\color{red}0.0029} & {\color{red}0.0066} & 33 \\
052 & {\color{red}0.0009} & > 1.0 & - & - & {\color{red}0.0019} & {\color{red}0.0002} & 33 \\
055 & {\color{red} < 0.0001} & {\color{red}0.0126} & - & - & 0.5134 & NA & 34 \\
058 & {\color{red} < 0.0001} & 0.8530 & 0.1518 & 0.1180 & - & - & 35 \\
059 & {\color{red} < 0.0001} & 0.2880 & {\color{red}0.0029} & {\color{red}0.0030} & {\color{red} < 0.0001} & {\color{red}0.0002} & 35 \\
068 & {\color{red} < 0.0001} & > 1.0 & {\color{red}0.0139} & {\color{red}0.0120} & {\color{red} < 0.0001} & {\color{red}0.0007} & 38 \\
078 & {\color{red} < 0.0001} & {\color{red}0.0033} & {\color{red} < 0.0001} & {\color{red}0.0012} & {\color{red}0.0029} & {\color{red}0.0001} & 41 \\
079 & {\color{red} < 0.0001} & {\color{red}0.0299} & - & - & 0.1198 & 0.0547 & 42 \\
080 & {\color{red} < 0.0001} & {\color{red}0.0149} & {\color{red}0.0009} & {\color{red}0.0008} & {\color{red} < 0.0001} & > 1.0 & 42 \\
083 & {\color{red} < 0.0001} & {\color{red}0.0022} & {\color{red}0.0029} & {\color{red}0.0005} & {\color{red} < 0.0001} & {\color{red}0.0009} & 43 \\
084 & {\color{red} < 0.0001} & {\color{red}0.0004} & {\color{red}0.0009} & > 1.0 & {\color{red}0.0079} & {\color{red}0.0123} & 43 \\
085 & {\color{red} < 0.0001} & {\color{red}0.0078} & {\color{red} < 0.0001} & {\color{red}0.0001} & {\color{red}0.0009} & {\color{red}0.0022} & 44 \\
087 & {\color{red} < 0.0001} & {\color{red}0.0089} & {\color{red}0.0189} & {\color{red}0.0112} & {\color{red}0.0149} & {\color{red}0.0171} & 44 \\
089 & {\color{red} < 0.0001} & {\color{red}0.0001} & {\color{red}0.0001} & {\color{red}0.0001} & {\color{red} < 0.0001} & > 1.0 & 45 \\
094 & {\color{red} < 0.0001} & {\color{red}0.1330} & {\color{red} < 0.0001} & {\color{red}0.0003} & {\color{red}0.0039} & {\color{red}0.0024} & 47 \\
096 & {\color{red} < 0.0001} & {\color{red}0.0001} & {\color{red} < 0.0001} & > 1.0 & {\color{red} < 0.0001} & > 1.0 & 48 \\
097 & {\color{red} < 0.0001} & {\color{red}0.0002} & {\color{red} < 0.0001} & {\color{red}0.0001} & {\color{red} < 0.0001} & > 1.0 & 48 \\
103 & {\color{red} < 0.0001} & 0.0724 & 0.0509 & 0.1133 & {\color{red}0.0279} & 0.0554 & 51 \\
109 & {\color{red} < 0.0001} & {\color{red} < 0.0001} & - & - & - & - & 53 \\
110 & {\color{red} < 0.0001} & {\color{red} < 0.0001} & {\color{red} < 0.0001} & 0.0635 & {\color{red} < 0.0001} & > 1.0 & 54 \\
111 & {\color{red} < 0.0001} & {\color{red}0.0001} & {\color{red}0.0019} & {\color{red}0.0010} & {\color{red}0.00099} & {\color{red}0.0023} & 54 \\
112 & {\color{red} < 0.0001} & {\color{red}0.4100} & {\color{red} < 0.0001} & {\color{red}0.0003} & {\color{red} < 0.0001} & {\color{red}0.0002} & 54 \\
113 & {\color{red} < 0.0001} & {\color{red} < 0.0001} & - & - & {\color{red}0.0069} & > 1.0 & 55 \\
115 & {\color{red} < 0.0001} & {\color{red} < 0.0001} & 0.1998 & NA & 0.2717 & 0.0964 & 56 \\
116 & {\color{red} < 0.0001} & {\color{red} < 0.0001} & 0.0549 & NA & 0.0539 & 0.0585 & 56 \\
121 & {\color{red} < 0.0001} & {\color{red} < 0.0001} & {\color{red} < 0.0001} & > 1.0 & {\color{red}0.0029} & > 1.0 & 59 \\
122 & {\color{red} < 0.0001} & {\color{red}0.0001} & {\color{red} < 0.0001} & {\color{red}0.0002} & {\color{red} < 0.0001} & > 1.0 & 59 \\
123 & {\color{red} < 0.0001} & {\color{red} < 0.0001} & - & - & {\color{red}0.0109} & {\color{red}0.0002} & 60 \\
126 & {\color{red} < 0.0001} & {\color{red} < 0.0001} & {\color{red} < 0.0001} & > 1.0 & {\color{red} < 0.0001} & > 1.0 & 61 \\
127 & {\color{red} < 0.0001} & {\color{red} < 0.0001} & {\color{red} < 0.0001} & 0.3410 & {\color{red}0.0019} & {\color{red}0.0008} & 62 \\
130 & {\color{red} < 0.0001} & {\color{red} < 0.0001} & {\color{red}0.0009} & {\color{red}0.0004} & {\color{red}0.0159} & {\color{red}0.0061} & 63 \\
131 & {\color{red} < 0.0001} & {\color{red}0.01410} & {\color{red} < 0.0001} & > 1.0 & 0.1448 & 0.0851 & 64 \\
132 & {\color{red} < 0.0001} & {\color{red}0.0001} & {\color{red}0.0029} & {\color{red}0.0001} & {\color{red}0.0009} & {\color{red}0.0021} & 64 \\
134 & {\color{red} < 0.0001} & {\color{red} < 0.0001} & {\color{red} < 0.0001} & {\color{red}0.0168} & {\color{red}0.0009} & {\color{red}0.0002} & 65 \\
135 & {\color{red} < 0.0001} & {\color{red} < 0.0001} & {\color{red}0.0069} & NA & {\color{red} < 0.0001} & > 1.0 & 66 \\
139 & {\color{red} < 0.0001} & {\color{red} < 0.0001} & {\color{red}0.0229} & {\color{red}0.0211} & {\color{red}0.0019} & {\color{red}0.0001} & 68 \\
144 & {\color{red} < 0.0001} & {\color{red}0.0036} & - & - & {\color{red}0.0209} & {\color{red}0.0033} & 71 \\
145 & {\color{red} < 0.0001} & {\color{red} < 0.0001} & {\color{red} < 0.0001} & > 1.0 & {\color{red} < 0.0001} & > 1.0 & 71 \\
147 & {\color{red} < 0.0001} & {\color{red} < 0.0001} & {\color{red} < 0.0001} & {\color{red}0.00813} & {\color{red}0.0119} & {\color{red}0.0059} & 72 \\
148 & {\color{red} < 0.0001} & {\color{red} < 0.0001} & {\color{red}0.0239} & {\color{red}0.0089} & - & - & 73 \\
153 & {\color{red} < 0.0001} & {\color{red} < 0.0001} & {\color{red} < 0.0001} & {\color{red}0.0282} & {\color{red} < 0.0001} & {\color{red}0.0003} & 76 \\
154 & {\color{red} < 0.0001} & {\color{red} < 0.0001} & {\color{red} < 0.0001} & {\color{red}0.0021} & {\color{red} < 0.0001} & 0.3820 & 77 \\
156 & {\color{red} < 0.0001} & {\color{red} < 0.0001} & {\color{red} < 0.0001} & > 1.0 & {\color{red}0.0009} & {\color{red}0.0005} & 78 \\
161 & {\color{red} < 0.0001} & {\color{red} < 0.0001} & 0.0979 & 0.0515 & - & - & 81 \\
162 & {\color{red} < 0.0001} & {\color{red} < 0.0001} & {\color{red}0.0069} & {\color{red}0.0168} & {\color{red}0.0439} & NA & 82 \\
164 & {\color{red} < 0.0001} & {\color{red} < 0.0001} & {\color{red} < 0.0001} & 0.4890 & - & - & 83 \\
165 & {\color{red} < 0.0001} & {\color{red} < 0.0001} & {\color{red} < 0.0001} & > 1.0 & {\color{red} < 0.0001} & {\color{red}0.0247} & 84 \\
166 & {\color{red} < 0.0001} & {\color{red} < 0.0001} & {\color{red} < 0.0001} & 0.1950 & {\color{red} < 0.0001} & > 1.0 & 84 \\
167 & {\color{red} < 0.0001} & {\color{red} < 0.0001} & {\color{red} < 0.0001} & {\color{red}0.0001} & {\color{red} < 0.0001} & 0.3990 & 85 \\
168 & {\color{red} < 0.0001} & {\color{red} < 0.0001} & {\color{red} < 0.0001} & 0.3290 & {\color{red} < 0.0001} & 0.2840 & 86 \\
169 & {\color{red} < 0.0001} & {\color{red} < 0.0001} & {\color{red}0.0069} & {\color{red}0.0106} & {\color{red} < 0.0001} & > 1.0 & 87 \\
170 & {\color{red} < 0.0001} & {\color{red} < 0.0001} & {\color{red} < 0.0001} & 0.9260 & {\color{red} < 0.0001} & > 1.0 & 87 \\
173 & {\color{red} < 0.0001} & {\color{red} < 0.0001} & {\color{red}0.0209} & {\color{red}0.0072} & {\color{red} < 0.0001} & > 1.0 & 89 \\
174 & {\color{red} < 0.0001} & {\color{red} < 0.0001} & {\color{red} < 0.0001} & 0.0533 & {\color{red} < 0.0001} & {\color{red}0.0005} & 90 \\
175 & {\color{red} < 0.0001} & {\color{red} < 0.0001} & {\color{red} < 0.0001} & {\color{red}0.0002} & {\color{red} < 0.0001} & 0.1710 & 91 \\
176 & {\color{red} < 0.0001} & {\color{red} < 0.0001} & {\color{red} < 0.0001} & 0.3140 & {\color{red} < 0.0001} & {\color{red}0.0048} & 92 \\
178 & {\color{red} < 0.0001} & {\color{red} < 0.0001} & {\color{red} < 0.0001} & {\color{red}0.0246} & {\color{red} < 0.0001} & > 1.0 & 93 \\
180 & {\color{red} < 0.0001} & {\color{red} < 0.0001} & {\color{red} < 0.0001} & 0.0665 & {\color{red} < 0.0001} & {\color{red}0.0014} & 95 \\
181 & {\color{red} < 0.0001} & {\color{red} < 0.0001} & {\color{red}0.0129} & {\color{red}0.0131} & {\color{red} < 0.0001} & {\color{red} < 0.0001} & 95 \\
182 & {\color{red} < 0.0001} & {\color{red} < 0.0001} & {\color{red} < 0.0001} & {\color{red}0.0073} & {\color{red} < 0.0001} & {\color{red}0.0021} & 96 \\
185 & {\color{red} < 0.0001} & {\color{red} < 0.0001} & {\color{red} < 0.0001} & 0.0607 & - & - & 99 \\
186 & {\color{red} < 0.0001} & {\color{red} < 0.0001} & {\color{red} < 0.0001} & {\color{red}0.0023} & {\color{red} < 0.0001} & {\color{red}0.0038} & 99 \\
187 & {\color{red} < 0.0001} & {\color{red} < 0.0001} & {\color{red}0.0029} & {\color{red}0.0028} & {\color{red} < 0.0001} & {\color{red}0.0039} & 100 \\
191 & {\color{red} < 0.0001} & {\color{red} < 0.0001} & {\color{red} < 0.0001} & {\color{red} < 0.0001} & {\color{red} < 0.0001} & {\color{red}0.0053} & 103 \\
192 & {\color{red} < 0.0001} & {\color{red} < 0.0001} & {\color{red} < 0.0001} & {\color{red}0.0002} & {\color{red} < 0.0001} & {\color{red} < 0.0001} & 104 \\
194 & {\color{red} < 0.0001} & {\color{red} < 0.0001} & {\color{red} < 0.0001} & 0.1160 & {\color{red} < 0.0001} & {\color{red}0.0001} & 106 \\
195 & {\color{red} < 0.0001} & {\color{red} < 0.0001} & {\color{red} < 0.0001} & {\color{red}0.0015} & {\color{red} < 0.0001} & {\color{red}0.0551} & 107 \\
198 & {\color{red} < 0.0001} & {\color{red} < 0.0001} & - & - & {\color{red} < 0.0001} & 0.7790 & 110 \\
199 & {\color{red} < 0.0001} & {\color{red} < 0.0001} & {\color{red} < 0.0001} & {\color{red} < 0.0001} & {\color{red} < 0.0001} & {\color{red}0.0083} & 110 \\
200 & {\color{red} < 0.0001} & {\color{red} < 0.0001} & {\color{red} < 0.0001} & {\color{red}0.0325} & {\color{red} < 0.0001} & {\color{red}0.0003} & 111 \\ \hline       
\label{tab:sorirotation1}
\end{longtable}
}

{\scriptsize
\begin{longtable}{cccccccc}
\caption{Teste Cramer e Hotelling aplicado na amostra I do catálogo III com rotação utilizando mediana.}
\label{tab:sorirotation2}
\hline
\multirow{2}{*}{\textbf{Cluster}} & \multicolumn{2}{|c|}{\textbf{Cenário 1}} & \multicolumn{2}{c|}{\textbf{Cenário 2}} & \multicolumn{2}{c|}{\textbf{Cenário 3}} & \multirow{2}{*}{\textbf{Nº galáxias}} \\ \cline{2-7}
                         & \multicolumn{1}{|c}{\textbf{Cramer}}       & \textbf{Hotelling}       & \textbf{Cramer}       & \textbf{Hotelling}       & \textbf{Cramer}       & \textbf{Hotelling}                              \\ \hline
001 & {\color{red}0.0289} & {\color{red}0.0017} & 0.0939 & {\color{red}0.0024} & - & - &  22 \\  
003 & {\color{red}0.0019} & {\color{red}0.0024} & {\color{red}0.0109} & {\color{red}0.0258} & {\color{red}0.0199} & 0.0577 & 22 \\
004 & {\color{red}0.0039} & {\color{red}0.0003} & 0.1098 & NA & {\color{red}0.0339} & {\color{red}0.0281} & 23 \\
006 & {\color{red}0.0009} & {\color{red}0.0009} & {\color{red}0.0099} & {\color{red}0.0153} & {\color{red}0.0059} & {\color{red}0.0275} & 23 \\
007 & {\color{red}0.0009} & > 1.0 & {\color{red}0.0099} & {\color{red}0.0150} & {\color{red}0.0059} & {\color{red}0.0033} & 23 \\
009 & {\color{red}0.0169} & {\color{red}0.0001} & 0.2917 & 0.2911 & 0.3356 & 0.0560 & 24 \\
010 & {\color{red}0.0019} & > 1.0 & {\color{red}0.0189} & {\color{red}0.0369} & {\color{red}0.0149} & {\color{red}0.0118} & 24 \\
011 & {\color{red}0.0009} & {\color{red}0.0002} & - & - &  {\color{red}0.0039} & {\color{red}0.0017} & 24 \\
012 & {\color{red} < 0.0001} & > 1.0 & {\color{red}0.0249} & {\color{red}0.0391} & {\color{red}0.0009} & {\color{red}0.0064} & 24 \\
013 & {\color{red} < 0.0001} & > 1.0 & 0.0019 & 0.0022 & 0.0429 & 0.0377 & 24 \\
014 & {\color{red} < 0.0001} & > 1.0 & 0.1718282 & NA & - & - & 25 \\
015 & {\color{red} < 0.0001} & > 1.0 & 0.3986 & NA & 0.4125 & 0.1685 & 25 \\
016 & {\color{red} < 0.0001} & > 1.0 & {\color{red}0.0159} & {\color{red}0.0052} & {\color{red}0.0069} & {\color{red}0.0026} & 25 \\
017 & {\color{red}0.0029} & {\color{red}0.0023} & {\color{red}0.0169} & {\color{red}0.0463} & {\color{red}0.0349} & 0.0574 & 25 \\
019 & {\color{red}0.0059} & {\color{red}0.0008} & 0.6523 & NA & 0.6313 & 0.5093 & 26 \\
020 & {\color{red}0.0009} & > 1.0 & {\color{red}0.0189} & {\color{red}0.0045} & {\color{red}0.0009} & {\color{red}0.0059} & 26 \\
021 & {\color{red} < 0.0001} & > 1.0 & {\color{red}0.0359} & {\color{red}0.0337} & 0.2087 & NA & 26 \\
023 & {\color{red} < 0.0001} & > 1.0 & 0.0269 & 0.0594 & 0.0099 & 0.0018 & 26 \\
024 & {\color{red}0.0009} & {\color{red}0.0005} & {\color{red}0.0329} & 0.0626 & {\color{red}0.0099} & {\color{red}0.0094} & 27 \\
025 & {\color{red}0.0009} & {\color{red}0.0017} & 0.0709 & 0.0846 & 0.1898 & 0.2670 & 27 \\
026 & {\color{red} < 0.0001} & > 1.0 & {\color{red}0.0459} & 0.1095 & {\color{red}0.0009} & {\color{red}0.0019} & 27 \\
027 & {\color{red}0.0009} & > 1.0 & {\color{red}0.0049} & {\color{red}0.0184} & {\color{red}0.0059} & {\color{red}0.0078} & 27 \\
028 & {\color{red}0.0019} & {\color{red}0.0001} & {\color{red}0.0139} & {\color{red}0.0011} & - & - &  28 \\
029 & {\color{red} < 0.0001} & > 1.0 & {\color{red}0.0009} & {\color{red}0.0013} & {\color{red}0.0389} & {\color{red}0.0343} & 28 \\
032 & {\color{red}0.0149} & > 1.0 & 0.2397 & 0.1458 & 0.5844 & NA & 28 \\
033 & {\color{red} < 0.0001} & {\color{red}0.0002} & 0.2397 & NA & {\color{red}0.0019} & {\color{red}0.0001} & 29 \\
034 & {\color{red} < 0.0001} & > 1.0 & {\color{red}0.0049} & {\color{red}0.0200} & {\color{red} < 0.0001} & {\color{red}0.0002} & 29 \\
035 & {\color{red} < 0.0001} & 0.5450 & {\color{red}0.0029} & {\color{red}0.0027} & {\color{red} < 0.0001} & {\color{red}0.0012} & 29 \\
036 & {\color{red} < 0.0001} & > 1.0 & {\color{red} < 0.0001} & {\color{red}0.0014} & {\color{red}0.0019} & {\color{red}0.0055} & 29 \\
037 & {\color{red} < 0.0001} & > 1.0 & {\color{red}0.0079} & {\color{red}0.0130} & {\color{red} < 0.0001} & {\color{red}0.0002} & 30 \\
040 & {\color{red} < 0.0001} & > 1.0 & {\color{red}0.0119} & {\color{red}0.0007} & {\color{red}0.0419} & {\color{red}0.0199} & 30 \\
041 & {\color{red} < 0.0001} & > 1.0 & {\color{red}0.0319} & {\color{red}0.0431} & - & - &  31 \\
042 & {\color{red} < 0.0001} & {\color{red}0.0006} & {\color{red}0.0359} & 0.0602 & {\color{red}0.0009} & {\color{red}0.0007} & 31 \\
044 & {\color{red}0.0009} & > 1.0 & - & - &  0.2207 & {\color{red}0.0403} & 31 \\
045 & {\color{red} < 0.0001} & 0.2630 & {\color{red}0.0029} & {\color{red}0.0004} & {\color{red}0.0059} & {\color{red}0.0036} & 32 \\
046 & {\color{red} < 0.0001} & 0.1090 & {\color{red} < 0.0001} & > 1.0 & {\color{red}0.0049} & {\color{red}0.0057} & 32 \\
047 & {\color{red}0.0009} & > 1.0 & 0.1588 & NA & 0.1908 & 0.2016 & 32 \\
049 & {\color{red} < 0.0001} & 0.0603 & - & - &  {\color{red}0.0029} & {\color{red}0.0004} & 33 \\
051 & {\color{red} < 0.0001} & > 1.0 & 0.1928 & 0.1026 & 0.1158 & 0.1753 & 33 \\
053 & {\color{red}0.0009} & > 1.0 & {\color{red}0.0259} & {\color{red}0.0167} & 0.0539 & {\color{red}0.0446} & 34 \\
054 & {\color{red} < 0.0001} & 0.3120 & {\color{red}0.0019} & {\color{red}0.0001} & {\color{red} < 0.0001} & > 1.0 & 34 \\
056 & {\color{red} < 0.0001} & > 1.0 & {\color{red} < 0.0001} & {\color{red}0.0004} & {\color{red}0.0009} & {\color{red}0.0021} & 35 \\
057 & {\color{red} < 0.0001} & 0.0577 & {\color{red}0.0129} & NA & {\color{red}0.0019} & {\color{red}0.0007} & 35 \\
060 & {\color{red} < 0.0001} & 0.0773 & {\color{red}0.0009} & {\color{red}0.0021} & {\color{red} < 0.0001} & {\color{red}0.0001} & 36 \\
061 & {\color{red}0.0019} & {\color{red}0.0138} & 0.1598 & {\color{red}0.0239} & 0.2827 & NA & 36 \\
062 & {\color{red} < 0.0001} & {\color{red}0.0094} & 0.0579 & {\color{red}0.0107} & - & - &  36 \\
063 & {\color{red} < 0.0001} & > 1.0 & 0.3376 & NA & {\color{red}0.0189} & {\color{red}0.0055} & 37 \\
064 & {\color{red} < 0.0001} & > 1.0 & {\color{red}0.0169} & {\color{red}0.0063} & {\color{red}0.0009} & {\color{red}0.0028} & 37 \\
065 & {\color{red} < 0.0001} & {\color{red}0.0096} & {\color{red}0.0199} & {\color{red}0.0050} & {\color{red} < 0.0001} & > 1.0 & 37 \\
066 & {\color{red} < 0.0001} & 0.9690 & {\color{red}0.0049} & {\color{red}0.0040} & {\color{red}0.0019} & {\color{red}0.0013} & 38 \\
067 & {\color{red}0.0009} & {\color{red}0.0001} & {\color{red}0.0069} & {\color{red}0.0115} & {\color{red}0.0079} & {\color{red}0.0234} & 38 \\
069 & {\color{red} < 0.0001} & {\color{red}0.0282} & {\color{red}0.0029} & {\color{red}0.0012} & {\color{red}0.0019} & {\color{red}0.0067} & 38 \\
070 & {\color{red} < 0.0001} & {\color{red}0.0119} & {\color{red} < 0.0001} & > 1.0 & {\color{red}0.0009} & {\color{red}0.0003} & 39 \\
071 & {\color{red} < 0.0001} & > 1.0 & {\color{red}0.0049} & {\color{red}0.0052} & {\color{red}0.0009} & {\color{red}0.0002} & 39 \\
072 & {\color{red} < 0.0001} & > 1.0 & {\color{red} < 0.0001} & {\color{red}0.0001} & {\color{red}0.0039} & {\color{red}0.0138} & 39 \\
073 & {\color{red} < 0.0001} & > 1.0 & 0.1018 & 0.1723 & 0.3106 & NA & 40 \\
074 & {\color{red} < 0.0001} & > 1.0 & {\color{red}0.0149} & {\color{red}0.0020} & - & - &  40 \\
075 & {\color{red} < 0.0001} & {\color{red}0.0028} & {\color{red}0.0069} & {\color{red}0.0025} & {\color{red} < 0.0001} & {\color{red}0.0001} & 40 \\
076 & {\color{red} < 0.0001} & 0.6060 & {\color{red}0.0009} & {\color{red}0.0042} & {\color{red} < 0.0001} & {\color{red}0.0001} & 41 \\
077 & {\color{red} < 0.0001} & {\color{red} < 0.0001} & {\color{red} < 0.0001} & > 1.0 & {\color{red}0.0019} & {\color{red}0.0009} & 41 \\
081 & {\color{red} < 0.0001} & 0.6860 & {\color{red}0.0059} & {\color{red}0.0038} & {\color{red} < 0.0001} & {\color{red}0.0008} & 42 \\
082 & {\color{red} < 0.0001} & 0.9540 & {\color{red}0.0009} & {\color{red}0.0002} & {\color{red}0.0019} & {\color{red}0.0023} & 43 \\
086 & {\color{red} < 0.0001} & {\color{red}0.0288} & {\color{red}0.0009} & {\color{red}0.0005} & {\color{red} < 0.0001} & {\color{red}0.0006} & 44 \\
088 & {\color{red} < 0.0001} & {\color{red}0.0194} & {\color{red} < 0.0001} & > 1.0 & {\color{red}0.0009} & {\color{red}0.0037} & 45 \\
090 & {\color{red} < 0.0001} & 0.4680 & {\color{red}0.0019} & {\color{red}0.0001} & - & - &  46 \\
091 & {\color{red} < 0.0001} & {\color{red}0.0001} & {\color{red} < 0.0001} & > 1.0 & {\color{red} < 0.0001} & {\color{red}0.0001} & 46 \\
092 & {\color{red} < 0.0001} & > 1.0 & {\color{red}0.0199} & {\color{red}0.0446} & {\color{red}0.0009} & {\color{red}0.0005} & 46 \\
093 & {\color{red} < 0.0001} & 0.3770 & {\color{red} < 0.0001} & {\color{red}0.0001} & {\color{red}0.1288} & 0.0982 & 47 \\
095 & {\color{red} < 0.0001} & {\color{red}0.0352} & {\color{red}0.0009} & {\color{red}0.0023} & {\color{red} < 0.0001} & > 1.0 & 47 \\
098 & {\color{red} < 0.0001} & {\color{red}0.0067} & 0.0869 & 0.1161 & 0.7172 & NA & 49 \\
099 & {\color{red} < 0.0001} & {\color{red}0.0087} & {\color{red}0.0169} & {\color{red}0.0133} & {\color{red} < 0.0001} & > 1.0 & 49 \\
100 & {\color{red} < 0.0001} & {\color{red}0.0020} & 0.0699 & 0.1894 & 0.0579 & 0.0561 & 49 \\
101 & {\color{red} < 0.0001} & 0.3450 & {\color{red}0.0079} & {\color{red}0.0014} & {\color{red}0.0099} & {\color{red}0.0122} & 50 \\
102 & {\color{red} < 0.0001} & 0.0514 & {\color{red}0.0019} & {\color{red}0.0010} & {\color{red}0.0029} & {\color{red}0.0083} & 50 \\
104 & {\color{red} < 0.0001} & 0.1700 & {\color{red}0.0089} & {\color{red}0.0012} & - & - &  51 \\
105 & {\color{red} < 0.0001} & {\color{red} < 0.0001} & {\color{red}0.0059} & {\color{red}0.0067} & {\color{red}0.0079} & {\color{red}0.0027} & 51 \\
106 & {\color{red} < 0.0001} & {\color{red} < 0.0001} & {\color{red}0.0009} & {\color{red}0.0025} & - & - &  52 \\
107 & {\color{red} < 0.0001} & {\color{red} < 0.0001} & {\color{red} < 0.0001} & {\color{red}0.0008} & {\color{red} < 0.0001} & > 1.0 & 52 \\
108 & {\color{red} < 0.0001} & {\color{red} < 0.0001} & - & - & {\color{red} < 0.0001} & > 1.0 & 53 \\
114 & {\color{red} < 0.0001} & {\color{red} < 0.0001} & {\color{red} < 0.0001} & 0.2470 & {\color{red} < 0.0001} & 0.0641 & 55 \\
117 & {\color{red} < 0.0001} & {\color{red}0.0051} & {\color{red} < 0.0001} & > 1.0 & {\color{red}0.0029} & {\color{red}0.0056} & 57 \\
118 & {\color{red} < 0.0001} & {\color{red} < 0.0001} & {\color{red} < 0.0001} & 0.7890 & {\color{red} < 0.0001} & {\color{red}0.0001} & 57 \\
119 & {\color{red} < 0.0001} & {\color{red} < 0.0001} & {\color{red} < 0.0001} & > 1.0 & {\color{red}0.0009} & > 1.0 & 58 \\
120 & {\color{red} < 0.0001} & {\color{red}0.0001} & {\color{red}0.0009} & {\color{red}0.0001} & {\color{red} < 0.0001} & > 1.0 & 58 \\
124 & {\color{red} < 0.0001} & {\color{red}0.0065} & {\color{red} < 0.0001} & > 1.0 & {\color{red}0.0419} & 0.0530 & 60 \\
125 & {\color{red} < 0.0001} & {\color{red}0.0001} & {\color{red} < 0.0001} & > 1.0 & {\color{red} < 0.0001} & 0.6960 & 61 \\
128 & {\color{red} < 0.0001} & {\color{red}0.0051} & {\color{red} < 0.0001} & > 1.0 & {\color{red}0.01098} & {\color{red}0.0048} & 62 \\
129 & {\color{red} < 0.0001} & {\color{red} < 0.0001} & {\color{red} < 0.0001} & > 1.0 & {\color{red} < 0.0001} & > 1.0 & 63 \\
133 & {\color{red} < 0.0001} & {\color{red}0.0318} & {\color{red} < 0.0001} & > 1.0 & 0.1838 & NA & 65 \\
136 & {\color{red} < 0.0001} & {\color{red} < 0.0001} & {\color{red} < 0.0001} & {\color{red}0.0001} & {\color{red} < 0.0001} & 0.0649 & 66 \\
137 & {\color{red} < 0.0001} & {\color{red}0.0001} & {\color{red} < 0.0001} & 0.1710 & {\color{red}0.0019} & {\color{red}0.0024} & 67 \\
138 & {\color{red} < 0.0001} & {\color{red}0.0009} & {\color{red} < 0.0001} & {\color{red}0.0006} & {\color{red} < 0.0001} & 0.3280 & 67 \\
140 & {\color{red} < 0.0001} & {\color{red} < 0.0001} & {\color{red} < 0.0001} & > 1.0 & {\color{red}0.0019} & {\color{red}0.0031} & 68 \\
141 & {\color{red} < 0.0001} & {\color{red} < 0.0001} & {\color{red} < 0.0001} & > 1.0 & {\color{red} < 0.0001} & {\color{red}0.0121} & 69 \\
142 & {\color{red} < 0.0001} & {\color{red} < 0.0001} & {\color{red}0.0469} & {\color{red}0.0425} & {\color{red} < 0.0001} & {\color{red}0.0002} & 70 \\
143 & {\color{red} < 0.0001} & {\color{red} < 0.0001} & {\color{red} < 0.0001} & 0.2040 & {\color{red} < 0.0001} & > 1.0 & 70 \\
146 & {\color{red} < 0.0001} & {\color{red} < 0.0001} & {\color{red} < 0.0001} & 0.2040 & {\color{red} < 0.0001} & > 1.0 & 72 \\
149 & {\color{red} < 0.0001} & {\color{red} < 0.0001} & {\color{red} < 0.0001} & 0.3980 & {\color{red}0.0009} & > 1.0 & 74 \\
150 & {\color{red} < 0.0001} & {\color{red} < 0.0001} & {\color{red} < 0.0001} & > 1.0 & {\color{red} < 0.0001} & 0.8110 & 74 \\
151 & {\color{red} < 0.0001} & {\color{red} < 0.0001} & {\color{red} < 0.0001} & 0.5680 & {\color{red} < 0.0001} & > 1.0 & 75 \\
152 & {\color{red} < 0.0001} & {\color{red} < 0.0001} & {\color{red} < 0.0001} & {\color{red} < 0.0001} & {\color{red} < 0.0001} & 0.1970 & 75 \\
155 & {\color{red} < 0.0001} & {\color{red} < 0.0001} & {\color{red} < 0.0001} & {\color{red}0.0001} & {\color{red}0.0029} & {\color{red}0.0015} & 77 \\
157 & {\color{red} < 0.0001} & {\color{red} < 0.0001} & {\color{red} < 0.0001} & > 1.0 & {\color{red} < 0.0001} & 0.2100 & 79 \\
158 & {\color{red} < 0.0001} & {\color{red} < 0.0001} & {\color{red} < 0.0001} & > 1.0 & {\color{red} < 0.0001} & 0.0727 & 79 \\
159 & {\color{red} < 0.0001} & {\color{red} < 0.0001} & {\color{red} < 0.0001} & {\color{red}0.0207} & {\color{red} < 0.0001} & 0.8910 & 80 \\
160 & {\color{red} < 0.0001} & {\color{red} < 0.0001} & {\color{red} < 0.0001} & > 1.0 & {\color{red}0.0009} & {\color{red}0.0004} & 80 \\
163 & {\color{red} < 0.0001} & {\color{red} < 0.0001} & {\color{red} < 0.0001} & {\color{red}0.0116} & {\color{red} < 0.0001} & 0.9310 & 82 \\
171 & {\color{red} < 0.0001} & {\color{red} < 0.0001} & {\color{red} < 0.0001} & > 1.0 & 0.4455544 & NA & 88 \\
172 & {\color{red} < 0.0001} & {\color{red} < 0.0001} & {\color{red} < 0.0001} & {\color{red} < 0.0001} & {\color{red} < 0.0001} & 0.1980 & 89 \\
177 & {\color{red} < 0.0001} & {\color{red} < 0.0001} & {\color{red} < 0.0001} & {\color{red}0.0446} & {\color{red} < 0.0001} & 0.5500 & 92 \\
179 & {\color{red} < 0.0001} & {\color{red} < 0.0001} & {\color{red} < 0.0001} & > 1.0 & {\color{red} < 0.0001} & {\color{red} < 0.0001} & 94 \\
183 & {\color{red} < 0.0001} & {\color{red} < 0.0001} & {\color{red} < 0.0001} & > 1.0 & {\color{red} < 0.0001} & > 1.0 & 97 \\
184 & {\color{red} < 0.0001} & {\color{red} < 0.0001} & 0.1498 & 0.1139 & {\color{red}0.0059} & {\color{red}0.0062} & 98 \\
188 & {\color{red} < 0.0001} & {\color{red} < 0.0001} & - & - &  {\color{red}0.0009} & > 1.0 & 101 \\
189 & {\color{red} < 0.0001} & {\color{red} < 0.0001} & {\color{red} < 0.0001} & > 1.0 & {\color{red} < 0.0001} & > 1.0 & 102 \\
190 & {\color{red} < 0.0001} & {\color{red} < 0.0001} & {\color{red} < 0.0001} & > 1.0 & {\color{red} < 0.0001} & {\color{red}0.0002} & 103 \\
193 & {\color{red} < 0.0001} & {\color{red} < 0.0001} & {\color{red} < 0.0001} & {\color{red}0.0001} & {\color{red} < 0.0001} & 0.0967 & 105 \\
196 & {\color{red} < 0.0001} & {\color{red} < 0.0001} & {\color{red} < 0.0001} & {\color{red}0.0076} & {\color{red}0} & {\color{red}0.0024} & 108 \\
197 & {\color{red} < 0.0001} & {\color{red} < 0.0001} & {\color{red} < 0.0001} & > 1.0 & {\color{red} < 0.0001} & {\color{red}0.0004} & 109 \\ \hline
\end{longtable}
}

 \begin{table}[H]
\caption{Teste Cramer e Hotelling aplicado na amostra II do catálogo III com rotação utilizando gap.}
\vspace{12pt}
\centering{}
\resizebox{1.0\textwidth}{!}{
\begin{tabular}{cccccccc}
\hline
\multirow{2}{*}{\textbf{Cluster}} & \multicolumn{2}{|c|}{\textbf{Cenário 1}} & \multicolumn{2}{c|}{\textbf{Cenário 2}} & \multicolumn{2}{c|}{\textbf{Cenário 3}} & \multirow{2}{*}{\textbf{Nº galáxias}} \\ \cline{2-7}
                         & \multicolumn{1}{|c}{\textbf{Cramer}}       & \textbf{Hotelling}       & \textbf{Cramer}       & \textbf{Hotelling}       & \textbf{Cramer}       & \textbf{Hotelling}       &                              \\ \hline
003 & 0.6643 & 0.9703 & {\color{red}0.0339} & NA & 0.1458 & 0.1320 & 22 \\ 
013 & {\color{red}0.0389} & {\color{red}0.0208} & 0.0579 & NA & 0.1988 & 0.3100 & 24 \\ 
019 & 0.5524 & 0.6272 & 0.6753 & 0.3431 & {\color{red}0.0429} & {\color{red}0.0182} & 26 \\ 
029 & 0.0889 & 0.1869 & {\color{red}0.0399} & 0.1120 & - & - &  28 \\ 
075 & 0.4285 & 0.5943 & 0.7682 & 0.7250 & {\color{red}0.0439} & {\color{red}0.0347} & 40 \\ 
076 & 0.6923 & 0.6958 & {\color{red}0.0459} & 0.1490 & 0.3946 & 0.8040 & 41 \\ 
077 & {\color{red}0.0089} & {\color{red}0.0051} & 0.0509 & 0.0828 & {\color{red}0.0499} & 0.1092 & 41 \\ 
084 & {\color{red}0.0349} & {\color{red}0.0401} & {\color{red}0.0049} & {\color{red}0.0211} & 0.1698 & 5.37E-02 & 43 \\ 
087 & 0.4175 & 0.5135 & {\color{red}0.0459} & NA & 0.5864 & NA & 44 \\ 
091 & 0.0769 & {\color{red}0.0405} & 0.7852 & 0.4170 & 0.0659 & 0.0846 & 46 \\ 
095 & {\color{red}0.0119} & {\color{red}0.0271} & {\color{red}0.0049} & 0.0654 & 0.4545 & 0.8071 & 47 \\ 
098 & {\color{red}0.0079} & {\color{red}0.0169} & 0.4945 & 0.5664 & {\color{red}0.0109} & 5.21E-02 & 49 \\ 
099 & 0.1568 & 0.2271 & 0.5794 & 0.7764 & {\color{red}0.0219} & {\color{red}0.0251} & 49 \\ 
104 & {\color{red}0.0379} & {\color{red}0.0450} & - & - &  0.1008 & 0.1361 & 51 \\ 
114 & 0.7572 & 0.7282 & 0.2417 & 0.4212 & 0.0709 & {\color{red}0.0398} & 55 \\ 
123 & 0.1518 & 0.1645 & 0.1448 & {\color{red}0.0082} & 0.4345 & NA & 60 \\ 
148 & {\color{red}0.0289} & {\color{red}0.0203} & 0.2227 & 0.4450 & - & - &  73 \\ 
152 & 0.0579 & {\color{red}0.04829} & {\color{red}0.0239} & 5.20E-02 & 0.9730 & 0.9445 & 75 \\ 
155 & {\color{red}0.0119} & {\color{red}0.0107} & 0.1518 & 1.52E-01 & {\color{red}0.0269} & 0.0585 & 77 \\ 
162 & 0.4685 & 0.5076 & 0.5174 & 0.7196 & {\color{red}0.0189} & {\color{red}0.0431} & 82 \\ 
182 & {\color{red}0.0419} & 0.07297 & 0.0539 & {\color{red}0.0470} & 0.7702 & 0.8008 & 96 \\ 
197 & {\color{red}0.0279} & {\color{red}0.0175} & 0.0719 & 0.1064 & 0.1208 & 0.1649 & 109 \\
198 & 0.0909 & 0.1341 & 0.9420 & 0.9737 & {\color{red} < 0.0001} & {\color{red}0.0003} & 110 \\ \hline		
\label{tab:sorinorotation1}
\end{tabular}
}
\end{table}


 \begin{table}[H]
\caption{Teste Cramer e Hotelling aplicado na amostra II do catálogo III com rotação utilizando mediana.}
\vspace{12pt}
\centering{}
\resizebox{1.0\textwidth}{!}{
\begin{tabular}{cccccccc}
\hline
\multirow{2}{*}{\textbf{Cluster}} & \multicolumn{2}{|c|}{\textbf{Cenário 1}} & \multicolumn{2}{c|}{\textbf{Cenário 2}} & \multicolumn{2}{c|}{\textbf{Cenário 3}} & \multirow{2}{*}{\textbf{Nº galáxias}} \\ \cline{2-7}
                         & \multicolumn{1}{|c}{\textbf{Cramer}}       & \textbf{Hotelling}       & \textbf{Cramer}       & \textbf{Hotelling}       & \textbf{Cramer}       & \textbf{Hotelling}       &                              \\ \hline
015 & 0.1178 & 0.2722 & {\color{red}0.0499} & NA & 0.2487 & NA & 25 \\
021 & {\color{red}0.0439} & 0.0710 & 0.0639 & 0.1537 & 0.2147 & 0.3745 & 26 \\ 
028 & 0.1558 & 0.7918 & 0.2547 & 0.2620 & {\color{red}0.0329} & {\color{red}0.0329} & 28 \\
039 & 0.7042 & 0.6370 & 0.5724 & 0.6585 & 0.1168 & {\color{red}0.0416} & 30 \\ \hline

\label{tab:sorinorotation2}
\end{tabular}
}
\end{table}

\textbf{MÉTODO DE HWANG \& LEE}

Aplicamos o método de \citeonline{hwang2007searching} para as duas amostras do catálogo III  e os resultados são apresentados nas tabelas \ref{tab:rotationhwang} e \ref{tab:norotationhwang}. Na amostra I, em 96.5\% dos casos foi dectectada rotação. Ou seja, o número
de falsos negativos (objetos com rotação que não apresentam indicação para tal) é pequeno, apenas 7 aglomerados teriam 
este diagnóstico. Por outro lado,
na amostra II, em 97.8\% dos casos houve indicação de algum grau de rotação, quando na verdade nenhum desses
sistemas rotaciona. Ou seja, o método de HL é "agressivo" para detectar rotação mas produz uma alta 
taxa de falsos positivos.

{\scriptsize
\begin{longtable}{ccccc}
\caption{Resultado do métdo Hwang \& Lee para a amostra I do catálogo III.}\label{tab:rotationhwang}
\\ \hline
\multicolumn{1}{l}{\textbf{Cluster}} & \multicolumn{1}{l}{\textbf{Velocidade Rotacional}} & \multicolumn{1}{l}{\textbf{Ângulo (radiano)}} & \multicolumn{1}{l}{\textbf{Ângulo (grau)}} \\ \hline
1 & -968.13 & 1.795 & 102.836 \\
2 & -1076.21 & 1.197 & 68.576 \\
3 & 761.15 & 1.346 & 77.112 \\
4 & -462.12 & 1.428 & 81.81 \\
5 & -358.95 & 1.285 & 73.618 \\
6 & -1081.13 & 0.286 & 16.385 \\
7 & -977.96 & 2.142 & 122.715 \\
8 & -565.29 & 2.713 & 155.428 \\
9 & -394.83 & 0.683 & 39.129 \\
10 & -888.25 & 1.776 & 101.747 \\
11 & -789.57 & 0.683 & 39.129 \\
12 & -493.52 & 3.005 & 172.156 \\
13 & 986.71 & 2.185 & 125.179 \\
14 & -711.44 & 1.702 & 97.508 \\
15 & -900.58 & 1.571 & 90.003 \\
16 & -995.15 & 0.654 & 37.468 \\
17 & -1089.72 & 2.618 & 149.985 \\
18 & 801.68 & 1.309 & 74.993 \\
19 & 722.24 & 2.011 & 115.21 \\
20 & 268.30 & 2.262 & 129.59 \\
21 & 903.82 & 0.754 & 43.197 \\
22 & -1002.72 & 2.262 & 129.59 \\
23 & -821.14 & 0.628 & 35.978 \\
24 & 998.10 & 0 & 0 \\
25 & 823.51 & 1.208 & 69.206 \\
26 & -573.22 & 2.417 & 138.47 \\
27 & - & - & - \\
28 & -1016.17 & 0.582 & 33.343 \\
29 & -511.79 & 2.211 & 126.668 \\
30 & -848.04 & 2.443 & 139.959 \\
31 & -595.85 & 0.233 & 13.349 \\
32 & 412.89 & 1.745 & 99.971 \\
33 & 761.15 & 1.01 & 57.863 \\
34 & 1004.33 & 1.459 & 83.586 \\
35 & -778.99 & 2.805 & 160.698 \\
36 & -1022.17 & 0.673 & 38.556 \\
37 & -714.70 & 1.192 & 68.29 \\
38 & -714.70 & 0.108 & 6.187 \\
39 & 850.60 & 0.867 & 49.67 \\
40 & 1007.13 & 1.733 & 99.284 \\
41 & 1009.74 & 0.733 & 41.994 \\
42 & -579.04 & 2.513 & 143.97 \\
43 & -1032.98 & 0 & 0 \\
44 & - & - & - \\
45 & -1184.3 & 1.419 & 81.295 \\
46 & -891.43 & 1.013 & 58.035 \\
47 & -891.43 & 1.824 & 104.497 \\
48 & 1085.40 & 1.317 & 75.451 \\
49 & -758.73 & 1.276 & 73.102 \\
50 & 943.54 & 1.178 & 67.488 \\
51 & 943.54 & 0.785 & 44.973 \\
52 & -900.58 & 3.043 & 174.333 \\
53 & -909.18 & 1.238 & 70.925 \\
54 & 879.06 & 1.047 & 59.983 \\
55 & -1046.74 & 1.047 & 59.983 \\
56 & -984.03 & 2.125 & 121.741 \\
57 & 885.13 & 1.386 & 79.404 \\
58 & -783.76 & 0.37 & 21.197 \\
59 & 951.88 & 0.554 & 31.739 \\
60 & -1119.45 & 1.346 & 77.112 \\
61 & -1054.60 & 1.077 & 61.701 \\
62 & -730.36 & 1.795 & 102.836 \\
63 & 1022.35 & 0.698 & 39.988 \\
64 & 1085.4 & 0.873 & 50.014 \\
65 & -932.11 & 1.484 & 85.018 \\
66 & -570.86 & 1.359 & 77.857 \\
67 & 901.37 & 1.783 & 102.148 \\
68 & -1184.3 & 0.849 & 48.639 \\
69 & 533.31 & 2.462 & 141.048 \\
70 & - & - & - \\
71 & -467.55 & 1.984 & 113.663 \\
72 & 965.94 & 1.405 & 80.492 \\
73 & -951.51 & 0.161 & 9.224 \\
74 & -1184.3 & 2.256 & 129.246 \\
75 & 1027.20 & 2.578 & 147.694 \\
76 & -1070.81 & 0.236 & 13.52 \\
77 & - & - & - \\
78 & 1085.4 & 2.749 & 157.49 \\
79 & -907.50 & 1.303 & 74.649 \\
80 & -962.86 & 2.452 & 140.475 \\
81 & 1085.4 & 1.839 & 105.356 \\
82 & 977.31 & 0.075 & 4.297 \\
83 & -1130.26 & 0.898 & 51.446 \\
84 & 977.31 & 2.543 & 145.688 \\
85 & -1131.51 & 1.315 & 75.336 \\
86 & 1085.4 & 2.484 & 142.308 \\
87 & -973.16 & 0.804 & 46.061 \\
88 & - & - & - \\
89 & -410.53 & 2.285 & 130.908 \\
90 & 1034.96 & 2.304 & 131.996 \\
91 & -982.54 & 1.047 & 59.983 \\
92 & -1184.3 & 1.536 & 87.997 \\
93 & -1134.95 & 0.888 & 50.874 \\
94 & 740.01 & 1.981 & 113.491 \\
95 & -1134.95 & 0.888 & 50.874 \\
96 & -1039.42 & 3.075 & 176.167 \\
97 & 988.81 & 1.805 & 103.408 \\
98 & 707.11 & 0.065 & 3.724 \\
99 & -947.87 & 2.356 & 134.975 \\
100 & 1085.4 & 1.963 & 112.46 \\
101 & 1085.4 & 1.667 & 95.502 \\
102 & -582.13 & 0.641 & 36.723 \\
103 & - & - & - \\
104 & 404.49 & 0.44 & 25.208 \\
105 & 858.43 & 2.262 & 129.59 \\
106 & -739.26 & 0.246 & 14.093 \\
107 & 284.32 & 2.279 & 130.564 \\
108 & 736.21 & 2.779 & 159.209 \\
109 & -1009.70 & 1.873 & 107.304 \\
110 & 956.92 & 2.608 & 149.412 \\
111 & 699.97 & 1.245 & 71.326 \\
112 & -1098.65 & 1.778 & 101.862 \\
113 & 538.99 & 0.465 & 26.64 \\
114 & 959.30 & 1.454 & 83.3 \\
115 & -606.55 & 1.542 & 88.341 \\
116 & -1143.03 & 1.257 & 72.014 \\
117 & -778.99 & 2.581 & 147.865 \\
118 & 801.68 & 1.178 & 67.488 \\
119 & -1104.66 & 1.929 & 110.512 \\
120 & -1064.84 & 1.929 & 110.512 \\
121 & -910.37 & 1.842 & 105.528 \\
122 & 928.86 & 1.137 & 65.139 \\
123 & -1184.3 & 1.757 & 100.659 \\
124 & -1030.42 & 1.757 & 100.659 \\
125 & -843.84 & 0.471 & 26.984 \\
126 & -881.67 & 2.88 & 164.995 \\
127 & -626.17 & 1.133 & 64.91 \\
128 & -1072.67 & 2.472 & 141.621 \\
129 & 829.14 & 1.723 & 98.711 \\
130 & -854.82 & 1.52 & 87.081 \\
131 & 256.77 & 2.942 & 168.547 \\
132 & -1184.3 & 1.297 & 74.305 \\
133 & -794.19 & 1.473 & 84.388 \\
134 & - & - & -  \\
135 & -1184.3 & 1.595 & 91.378 \\
136 & -1079.54 & 2.223 & 127.356 \\
137 & 810.28 & 3.094 & 177.255 \\
138 & -977.96 & 1.761 & 100.888 \\
139 & 543.38 & 2.954 & 169.235 \\
140 & 1085.4 & 1.688 & 96.706 \\
141 & -1050.78 & 2.033 & 116.471 \\
142 & -789.57 & 1.002 & 57.405 \\
143 & -954.04 & 2.641 & 151.303 \\
144 & -924.90 & 0.449 & 25.723 \\
145 & 663.88 & 1.391 & 79.69 \\
146 & -1120.36 & 1.106 & 63.363 \\
147 & 765.72 & 2.611 & 149.584 \\
148 & 896.25 & 0.654 & 37.468 \\
149 & -997.74 & 1.205 & 69.034 \\
150 & 1023.21 & 2.625 & 150.386 \\
151 & 563.98 & 1.189 & 68.118 \\
152 & 993.38 & 1.613 & 92.409 \\
153 & -1002.72 & 2.974 & 170.38 \\
154 & 816.62 & 1.033 & 59.181 \\
155 & 846.48 & 0.413 & 23.661 \\
156 & 1085.4 & 1.306 & 74.821 \\
157 & 794.41 & 1.208 & 69.206 \\
158 & -1038.80 & 1.087 & 62.274 \\
159 & -1126.83 & 2.903 & 166.313 \\
160 & 1027.93 & 0.597 & 34.202 \\
161 & -1042.44 & 0.393 & 22.515 \\
162 & 1085.4 & 1.823 & 104.44 \\
163 & -1184.3 & 0.698 & 39.988 \\
164 & -1101.26 & 2.26 & 129.475 \\
165 & 811.94 & 2.422 & 138.756 \\
166 & -1047.57 & 1.211 & 69.378 \\
167 & 923.27 & 1.197 & 68.576 \\
168 & -1104.19 & 0.739 & 42.337 \\
169 & -1131.51 & 1.096 & 62.79 \\
170 & -920.38 & 1.242 & 71.154 \\
171 & 954.95 & 2.492 & 142.767 \\
172 & 1033.81 & 2.035 & 116.585 \\
173 & -591.08 & 0.321 & 18.39 \\
174 & -1082.29 & 2.895 & 165.855 \\
175 & 681.89 & 0.559 & 32.025 \\
176 & -984.76 & 2.693 & 154.282 \\
177 & -1034.64 & 1.761 & 100.888 \\
178 & -1134.95 & 2.595 & 148.668 \\
179 & -940.24 & 1.757 & 100.659 \\
180 & -1063.57 & 2.005 & 114.866 \\
181 & -942.84 & 2.908 & 166.599 \\
182 & 1061.50 & 1.058 & 60.613 \\
183 & -1113.37 & 1.734 & 99.341 \\
184 & -997.10 & 0.713 & 40.848 \\
185 & -1022.17 & 0.481 & 27.556 \\
186 & -836.89 & 1.988 & 113.893 \\
187 & -725.77 & 1.142 & 65.425 \\
188 & 994.61 & 1.445 & 82.784 \\
189 & 1085.4 & 3.110 & 178.172 \\
190 & 729.36 & 0.462 & 26.468 \\
191 & -1117.54 & 0.431 & 24.692 \\
192 & -765.61 & 0.641 & 36.723 \\
193 & -900.58 & 2.658 & 152.277 \\
194 & -622.27 & 2.932 & 167.974 \\
195 & 1063.98 & 2.015 & 115.439 \\
196 & 1042.97 & 1.674 & 95.903 \\
197 & -595.85 & 1.28 & 73.331 \\
198 & -955.24 & 1.499 & 85.878 \\
199 & 877.17 & 2.623 & 150.272 \\
200 & 672.72 & 1.342 & 76.883 \\ \hline
\end{longtable}
}

{\scriptsize
\begin{longtable}{ccccc}
\caption{Resultado do métdo Hwang \& Lee para a amostra II do catálogo III.}\label{tab:norotationhwang}
\hline
\multicolumn{1}{l}{\textbf{Cluster}} & \multicolumn{1}{l}{\textbf{Velocidade Rotacional}} & \multicolumn{1}{l}{\textbf{Ângulo (radiano)}} & \multicolumn{1}{l}{\textbf{Ângulo (grau)}} \\ \hline
1 & 761.15 & 2.394 & 137.152 \\
2 & 112.67 & 0 & 0 \\
3 & 869.23 & 0.748 & 42.853 \\
4 & 156.88 & 0.143 & 8.192 \\
5 & 569.55 & 0.428 & 24.52 \\
6 & 879.06 & 1.856 & 106.33 \\
7 & -1081.13 & 2.57 & 147.235 \\
8 & 672.72 & 0 & 0 \\
9 & 591.98 & 2.868 & 164.308 \\
10 & 591.98 & 2.049 & 117.387 \\
11 & -98.79 & 1.093 & 62.618 \\
12 & 493.30 & 0 & 0 \\
13 & 493.30 & 2.185 & 125.179 \\
14 & -427.73 & 0.524 & 30.02 \\
15 & 990.82 & 2.749 & 157.49 \\
16 & 423.40 & 2.487 & 142.48 \\
17 & 707.11 & 0.524 & 30.02 \\
18 & -238.59 & 0.131 & 7.505 \\
19 & 631.46 & 0.251 & 14.38 \\
20 & 813.03 & 0.377 & 21.598 \\
21 & -367.20 & 1.759 & 100.773 \\
22 & 722.24 & 1.005 & 57.576 \\
23 & -911.93 & 1.759 & 100.773 \\
24 & 648.91 & 2.296 & 131.538 \\
25 & 736.21 & 2.054 & 117.674 \\
26 & 561.62 & 0.121 & 6.932 \\
27 & -398.63 & 1.087 & 62.274 \\
28 & 412.89 & 2.909 & 166.657 \\
29 & -763.98 & 1.396 & 79.977 \\
30 & -848.04 & 1.164 & 66.686 \\
31 & -595.85 & 1.047 & 59.983 \\
32 & -511.79 & 2.793 & 160.011 \\
33 & -616.87 & 0.112 & 6.416 \\
34 & -616.87 & 1.459 & 83.586 \\
35 & -535.81 & 1.346 & 77.112 \\
36 & -535.81 & 0.112 & 6.416 \\
37 & -558.17 & 2.6 & 148.954 \\
38 & -558.17 & 1.3 & 74.477 \\
39 & -871.23 & 0.217 & 12.432 \\
40 & -714.70 & 2.275 & 130.335 \\
41 & -579.04 & 0.524 & 30.02 \\
42 & 707.11 & 0.838 & 48.009 \\
43 & -427.73 & 1.571 & 90.003 \\
44 & -730.36 & 2.094 & 119.965 \\
45 & 865.75 & 0.912 & 52.248 \\
46 & -671.78 & 2.027 & 116.127 \\
47 & 572.88 & 1.621 & 92.867 \\
48 & -598.57 & 0.507 & 29.046 \\
49 & 588.90 & 2.749 & 157.49 \\
50 & 943.54 & 1.571 & 90.003 \\
51 & -900.58 & 2.651 & 151.876 \\
52 & -475.01 & 2.651 & 151.876 \\
53 & -1046.74 & 1.142 & 65.425 \\
54 & -977.96 & 0.286 & 16.385 \\
55 & -496.51 & 0.286 & 16.385 \\
56 & -583.49 & 2.218 & 127.069 \\
57 & -850.52 & 2.772 & 158.808 \\
58 & - & - & -  \\ 
59 & 885.13 & 0.277 & 15.869 \\
60 & -17.02 & 1.257 & 72.014 \\
61 & 566.61 & 1.077 & 61.701 \\
62 & 112.67 & 1.795 & 102.836 \\
63 & 896.25 & 3.054 & 174.964 \\
64 & 707.11 & 0.96 & 54.998 \\
65 & 770.16 & 0.087 & 4.984 \\
66 & 717.34 & 0.425 & 24.348 \\
67 & -1122.95 & 2.802 & 160.527 \\
68 & 717.34 & 0.764 & 43.77 \\
69 & 1085.4 & 1.189 & 68.118 \\
70 & -587.01 & 2.398 & 137.381 \\
71 & 786.75 & 1.075 & 61.587 \\
72 & -646.73 & 0.248 & 14.208 \\
73 & -1067.90 & 1.208 & 69.206 \\
74 & -835.11 & 1.208 & 69.206 \\
75 & -1067.90 & 1.208 & 69.206 \\
76 & -900.58 & 0.707 & 40.504 \\
77 & -673.61 & 2.278 & 130.507 \\
78 & 858.43 & 1.414 & 81.008 \\
79 & 863.96 & 0.919 & 52.65 \\
80 & -907.50 & 1.609 & 92.18 \\
81 & -464.63 & 2.452 & 140.475 \\
82 & -914.09 & 1.122 & 64.279 \\
83 & 274.79 & 1.496 & 85.706 \\
84 & -1184.3 & 2.693 & 154.282 \\
85 & 768.69 & 1.461 & 83.701 \\
86 & 927.04 & 3.069 & 175.823 \\
87 & 1085.4 & 2.192 & 125.58 \\
88 & -823.21 & 1.999 & 114.523 \\
89 & -1132.71 & 0.143 & 8.192 \\
90 & 984.52 & 1.466 & 83.987 \\
91 & 480.14 & 2.094 & 119.965 \\
92 & -326.85 & 0.279 & 15.984 \\
93 & -1184.3 & 0.068 & 3.896 \\
94 & -740.22 & 1.776 & 101.747 \\
95 & -641.54 & 1.434 & 82.154 \\
96 & -653.09 & 1.337 & 76.597 \\
97 & 795.65 & 0.802 & 45.947 \\
98 & -806.01 & 2.225 & 127.47 \\
99 & -1184.3 & 0 & 0 \\
100 & 1038.11 & 2.88 & 164.995 \\
101 & 1039.08 & 1.026 & 58.78 \\
102 & 575.87 & 2.244 & 128.559 \\
103 & 813.03 & 0.817 & 46.806 \\
104 & 631.46 & 2.073 & 118.762 \\
105 & 1040.01 & 2.639 & 151.188 \\
106 & 907.38 & 2.156 & 123.517 \\
107 & 862.88 & 2.341 & 134.116 \\
108 & 736.21 & 0.906 & 51.905 \\
109 & -311.33 & 0.423 & 24.234 \\
110 & 785.62 & 2.904 & 166.37 \\
111 & -413.45 & 1.838 & 105.299 \\
112 & -884.52 & 2.727 & 156.23 \\
113 & -1184.3 & 2.851 & 163.334 \\
114 & 202.73 & 2.094 & 119.965 \\
115 & -1019.23 & 1.428 & 81.81 \\
116 & 755.26 & 0.571 & 32.713 \\
117 & - & - & - \\ 
118 & 234.26 & 2.581 & 147.865 \\
119 & 727.02 & 3.086 & 176.797 \\
120 & 567.74 & 1.157 & 66.285 \\
121 & 694.07 & 1.517 & 86.909 \\
122 & 615.80 & 1.842 & 105.528 \\
123 & -1068.89 & 1.225 & 70.18 \\
124 & 931.52 & 1.651 & 94.586 \\
125 & - & - & -  \\ 
126 & -730.36 & 0.576 & 32.999 \\
127 & 713.31 & 0.567 & 32.483 \\
128 & 1085.4 & 2.987 & 171.125 \\
129 & -86.05 & 2.838 & 162.589 \\
130 & - & - & -  \\ 
131 & 544.99 & 0.598 & 34.259 \\
132 & 581.02 & 1.496 & 85.706 \\
133 & -936.05 & 0.049 & 2.807 \\
134 & 305.19 & 1.718 & 98.424 \\
135 & -904.95 & 0.918 & 52.592 \\
136 & -730.36 & 1.063 & 60.899 \\
137 & 535.17 & 2.618 & 149.985 \\
138 & 810.28 & 2.618 & 149.985 \\
139 & 577.25 & 2.72 & 155.829 \\
140 & 678.88 & 1.219 & 69.837 \\
141 & -516.74 & 1.663 & 95.273 \\
142 & -625.09 & 0.501 & 28.702 \\
143 & -1151.40 & 2.094 & 119.965 \\
144 & -243.99 & 2.693 & 154.282 \\
145 & -373.69 & 2.872 & 164.537 \\
146 & -1152.33 & 0.177 & 10.14 \\
147 & -481.01 & 0.133 & 7.62 \\
148 & -1058.20 & 0.087 & 4.984 \\
149 & -997.74 & 0.646 & 37.009 \\
150 & -966.65 & 0.904 & 51.79 \\
151 & -540.19 & 3.142 & 180.005 \\
152 & 502.63 & 0.17 & 9.739 \\
153 & 510.40 & 2.555 & 146.376 \\
154 & 308.92 & 0.413 & 23.661 \\
155 & -1094.70 & 1.901 & 108.908 \\
156 & -742.15 & 2.53 & 144.944 \\
157 & 1027.20 & 2.94 & 168.433 \\
158 & -1038.80 & 0.161 & 9.224 \\
159 & -293.65 & 0.676 & 38.728 \\
160 & -207.46 & 1.074 & 61.529 \\
161 & -701.98 & 0.903 & 51.733 \\
162 & -651.90 & 1.357 & 77.743 \\
163 & -1100.23 & 0.543 & 31.108 \\
164 & 974.68 & 2.337 & 133.887 \\
165 & -965.53 & 0.151 & 8.651 \\
166 & -1047.57 & 0.189 & 10.828 \\
167 & -914.09 & 1.309 & 74.993 \\
168 & 764.97 & 0.591 & 33.858 \\
169 & 979.83 & 2.192 & 125.58 \\
170 & -550.89 & 2.228 & 127.642 \\
171 & 746.24 & 1.878 & 107.591 \\
172 & 827.48 & 2.535 & 145.23 \\
173 & -384.74 & 2.32 & 132.913 \\
174 & 906.88 & 0.706 & 40.447 \\
175 & -1159.08 & 2.653 & 151.99 \\
176 & 935.74 & 3.073 & 176.052 \\
177 & -1134.41 & 2.279 & 130.564 \\
178 & -1184.3 & 1.024 & 58.665 \\
179 & -452.13 & 1.216 & 69.665 \\
180 & 747.36 & 0.769 & 44.056 \\
181 & 892.23 & 1.036 & 59.352 \\
182 & 774.80 & 2.282 & 130.736 \\
183 & 801.68 & 0.458 & 26.239 \\
184 & 594.02 & 1.393 & 79.805 \\
185 & -1045.33 & 1.346 & 77.112 \\
186 & -489.49 & 1.411 & 80.836 \\
187 & -404.80 & 1.301 & 74.534 \\
188 & -1116.20 & 1.759 & 100.773 \\
189 & 141.56 & 2.706 & 155.027 \\
190 & -984.03 & 0.277 & 15.869 \\
191 & -717.00 & 1.448 & 82.956 \\
192 & 909.11 & 1.068 & 61.186 \\
193 & 452.50 & 0.423 & 24.234 \\
194 & -989.75 & 0 & 0 \\
195 & -756.05 & 0.089 & 5.099 \\
196 & -887.33 & 0.352 & 20.166 \\
197 & -301.63 & 2.88 & 164.995 \\
198 & -1163.47 & 1.499 & 85.878 \\
199 & -1017.71 & 2.45 & 140.361 \\
200 & 713.99 & 1.085 & 62.16 \\ \hline
\end{longtable}
}

\section{Correção da massa do aglomerado}

Para o cálculo de correção da massa adotamos a sequência definida por \citeonline{manolopoulou2016galaxy}. Segundo \citeonline{manolopoulou2016galaxy} para a maioria dos aglomerados rotativos a velocidade de dispersão corrigida não é muito alterada, porém as alterações são relevantes. A correção é dada pela diferença fracionária entre a velocidade de dispersão corrigida do aglomerado e a não corrigida, definida como:

\begin{equation}
\delta \sigma_{v} = \frac{\sigma_{v} - \sigma_{v,cor}}{\sigma_{v}},
\label{eq:corrveldisp}
\end{equation}

\noindent e encontramos o valor da massa corrigida utilizando a fórmula \ref{eq:corrmass}

\begin{equation}
M_{cor} \simeq M (1 - \delta \sigma_{v})^2.
\label{eq:corrmass}
\end{equation}


\noindent Ou podemos realizar o cálculo simplesmente pelo Teorema do Virial dado por: 

\begin{equation}
M = \frac{\sigma^2 R^2}{G},
\label{eq:massa}
\end{equation}

\noindent onde $\omega$ é a velocidade rotacional, $R$ é o raio, $M$ a massa e $G$ é a constante gravitacional, e sigma deve ser
corrigida por $\delta \sigma_{v}$.

Os resultados para os três catálogos são exibidos nas tabelas \ref{tab:massselec20} a \ref{tab:massamostrai}. 
Vemos na tabela \ref{tab:massselec20} que 7 dos 11 aglomerados (64\%) com indicação de rotação 
possuem  negativo, o que implica em um aumento de sua massa após a correção dada pela equação \ref{cormass}.
Em apenas 4 dos 11 casos (36\%) ocorre redução da massa. Notem que as variações de massa são significativas, indo de 15.89\% a 28.0\%.
No caso da amostra do NoSocs, tabela \ref{massnosocs}, encontramos um resultado inverso: em 18 dos 47 aglomerados (38\%) 
vemos um aumento da massa dos aglomerados, enquanto em 30 sistemas (62\%) obtivemos uma redução da massa após
a correção. Notemos, mais uma vez, que as variações de massa são significativas, indo de de 13.48\% a 82.64\%.
Finalmente, para todos os aglomerados da amostra I do catálogo III vemos que $\delta\sigma$ é negativo, implicando
em aumento da massa corrigida. Uma vez que esta amostra corresponde a sistemas isolados, isto pode indicar que a rotação nestes casos sempre leva a um aumento da massa virial.

É importante destacar que para o cálculo de correção da massa estamos considerando apenas a componente rotacional como contribuinte para o ajuste, não analisamos outras possíveis fontes, como desvios do próprio equilíbrio virial. Fazemos ainda a observação de que
a correção não se aplica ao conjunto de dados da amostra II do catálogo III, uma vez que esta amostra foi definida sem rotação.

{\scriptsize
\begin{longtable}{cccccc}
\caption{Cálculo correção da massa para o catálogo selec20.}
\label{tab:massselec20}
\\ \hline
\textbf{Cluster} & \textbf{Anderson-Darling p-value} & \textbf{Vrot ($km \, s^{-1}$)} & \textbf{$\delta \sigma$} & \textbf{M200 ($10^{14}  M\odot$)} & \textbf{M_{cor} ($10^{14}  M\odot$)} \\ \hline
2 & {\color{red}0.0175} & 1014.69 & -0.06 & 6.7 & 7.59 \\
4 & 0.5148 & 816.91 & 0.15 & 7.42 & 5.34 \\
5 & {\color{red}0.0254} & 1284.33 & -0.19 & 14.98 & 21.07 \\
7 & {\color{red}0.0005} & 1878.89 & -0.59 & 23.12 & 58.21 \\
8 & {\color{red}0.0003} & 824.97 & 0.08 & 4.8 & 4.04 \\
9 & {\color{red}0.0021} & 837.38 & 0.1 & 5.88 & 4.77 \\
10 & {\color{red}0.0029} & 1232.28 & -0.08 & 17.91 & 20.93 \\
11 & {\color{red}0.0001} & 1154.06 & -0.13 & 10.24 & 13.14 \\
12 & {\color{red}0.0001} & 2728.77 & -0.99 & 53.6 & 212.33 \\
16 & 0.2477 & 781.13 & 0.12 & 4.19 & 3.26 \\
17 & {\color{red}< 0.0001} & 2239.97 & -0.82 & 31.28 & 103.96 \\ \hline
\end{longtable}
}

{\scriptsize
\begin{longtable}{cccccc}
\caption{Cálculo correção da massa para o catálogo NoSocs.}
\label{tab:massnosocs}
\\ \hline
\textbf{Cluster} & \textbf{Anderson-Darling p-value} & \textbf{Vrot ($km \, s^{-1}$)} & \textbf{$\Delta \sigma$} &\textbf{M200 ($10^{14}  M\odot$)} & \textbf{M_{cor} ($10^{14}  M\odot$)}\\ \hline
01238 & {\color{red}0.0042} & 797.69 & 0.09 & 3.84 & 3.15 \\
01831 & 0.1275 & 345.55 & 0.51 & 0.99 & 0.24 \\
01836 & 0.4749 & 448.87 & 0.42 & 1.84 & 0.62 \\
02433 & 0.2596 & 593.83 & 0.17 & 1.13 & 0.77 \\
02440 & 0.8344 & 302.64 & 0.58 & 1.21 & 0.21 \\
03176 & 0.5617 & 428.81 & 0.41 & 1.2 & 0.42 \\
03907 & 0.4525 & 547.83 & 0.30 & 1.86 & 0.92 \\
04470 & 0.4322 & 775.23 & 0.08 & 3.01 & 2.55 \\
04681 & 0.5913 & 1084.22 & -0.07 & 9.07 & 10.4 \\
04703 & {\color{red}0.0016} & 1147.11 & -0.19 & 6.57 & 9.37 \\
05359 & 0.9101 & 656.00 & 0.2 & 2.54 & 1.64 \\
05447 & 0.2530 & 497.66 & 0.36 & 1.82 & 0.75 \\
06256 & 0.2112 & 665.03 & 0.21 & 3.09 & 1.93 \\
06286 & {\color{red}0.0431} & 673.53 & 0.2 & 3.08 & 1.95 \\
06547 & {\color{red}0.0012} & 1104.74 & -0.16 & 6.36 & 8.49 \\
07703 & 0.2791 & 308.04 & 0.56 & 1.02 & 0.2 \\
08022 & 0.5450 & 650.79 & 0.18 & 2.19 & 1.47 \\
08721 & 0.8282 & 390.67 & 0.49 & 1.74 & 0.45 \\
10006 & 0.8143 & 932.21 & -0.02 & 5.02 & 5.24 \\
10008 & {\color{red}0.0017} & 1524.44 & -0.59 & 6.81 & 17.32 \\
10013 & 0.7772 & 933.65 & -0.06 & 4.05 & 4.56 \\
10015 & {\color{red}0.0146} & 1291.87 & -0.34 & 6.91 & 12.44 \\
10016 & {\color{red}0.0097} & 666.71 & 0.24 & 3.83 & 2.23 \\
10018 & 0.5140 & 748.49 & 0.14 & 3.67 & 2.7 \\
10020 & {\color{red}0.0039} & 1195.4 & -0.15 & 10.42 & 13.83 \\
10023 & {\color{red}0.0007} & 1738.54 & -0.60 & 14.45 & 36.86 \\
10024 & {\color{red}0.0008} & 1873.25 & -0.69 & 16.18 & 46.23 \\
10027 & 0.3834 & 824.85 & 0.1 & 5.11 & 4.1 \\
10028 & {\color{red}0.0025} & 830.93 & 0.0 & 2.83 & 2.83 \\
10029 & {\color{red}0.0040} & 505.62 & 0.35 & 2.04 & 0.87 \\
10030 & 0.6276 & 323.01 & 0.54 & 1.13 & 0.24 \\
10031 & {\color{red}0.0086} & 1070.37 & -0.05 & 9.46 & 10.34 \\
10035 & 0.4526 & 807.11 & 0.1 & 4.12 & 3.36 \\
10036 & 0.4284 & 647.85 & 0.18 & 2.27 & 1.53 \\
10037 & {\color{red}0.0122} & 1141.23 & -0.11 & 9.92 & 12.12 \\
10040 & 0.1201 & 635.65 & 0.24 & 3.08 & 1.76 \\
10043 & {\color{red}0.0102} & 686.59 & 0.17 & 3.04 & 2.09 \\
10044 & {\color{red}< 0.0001} & 1431.5 & -0.39 & 9.82 & 19.06 \\
10045 & {\color{red}0.0019} & 450.57 & 0.33 & 0.77 & 0.35 \\
10050 & 0.0944 & 795.6 & 0.04 & 2.87 & 2.64 \\
10051 & {\color{red}< 0.0001} & 857.33 & 0.07 & 5.64 & 4.88 \\
10053 & {\color{red}< 0.0001} & 981.84 & -0.03 & 6.93 & 7.34 \\
10054 & {\color{red}< 0.0001} & 1230.39 & -0.22 & 9.81 & 14.51 \\
10056 & 0.0748 & 1457.95 & -0.34 & 12.99 & 23.49 \\
10060 & 0.2142 & 803.62 & 0.10 & 4.31 & 3.52 \\
10063 & {\color{red}0.0001} & 1236.38 & -0.35 & 5.26 & 9.56 \\
10064 & 0.4375 & 1122.04 & -0.12 & 8.36 & 10.57 \\ \hline
\end{longtable}
}

{\scriptsize
\begin{longtable}{cccccc}
\caption{Cálculo correção da massa para o catálogo III - amostra I.}
\label{tab:massamostrai}
\\ \hline
\textbf{Cluster} & \textbf{Anderson-Darling p-value} &  \textbf{vrot ($km \, s^{-1}$)} & \textbf{$\Delta \sigma$} &\textbf{M200 ($10^{14}  M\odot$)} & \textbf{M_{cor} ($10^{14}  M\odot$)}\\ \hline
14 & {\color{red}0.0293} & 844.81 & -0.17 & 1.25 & 1.72 \\
15 & 0.0577 & 873.67 & -0.20 & 1.27 & 1.85 \\
25 & 0.1552 & 769.92 & -0.05 & 1.51 & 1.66 \\
28 & {\color{red}0.0325} & 1046.24 & -0.37 & 1.59 & 3.00 \\
40 & 0.0700 & 1012.17 & -0.30 & 1.96 & 3.30 \\
41 & {\color{red}0.0046} & 1186.46 & -0.51 & 2 & 4.54 \\
47 & {\color{red}0.0315} & 883.96 & -0.10 & 2.21 & 2.70 \\
53 & {\color{red}0.0094} & 1235.94 & -0.51 & 2.45 & 5.61 \\
55 & {\color{red}0.0001} & 1430.23 & -0.73 & 2.54 & 7.64 \\
62 & {\color{red}0.0022} & 1270.69 & -0.54 & 2.87 & 6.77 \\
63 & 0.0801 & 1079.88 & -0.29 & 2.92 & 4.82 \\
69 & {\color{red}0.0153} & 1404.31 & -0.67 & 3.24 & 9.06 \\
73 & {\color{red}0.0061} & 1229.3 & -0.46 & 3.47 & 7.38 \\
74 & 0.0592 & 943.69 & -0.10 & 3.53 & 4.26 \\
79 & {\color{red}0.0080} & 1056.99 & -0.22 & 3.85 & 5.71 \\
87 & {\color{red}0.0097} & 1051.61 & -0.17 & 4.42 & 6.10 \\
90 & {\color{red}0.0179} & 1328.09 & -0.45 & 4.65 & 9.84 \\
98 & {\color{red}> 0.0001} & 1058.56 & -0.16 & 5.34 & 7.22 \\
100 & {\color{red}0.0001} & 1485.14 & -0.63 & 5.53 & 14.62 \\
104 & {\color{red}0.0109} & 1514.65 & -0.64 & 5.92 & 15.87 \\
105 & {\color{red}> 0.0001} & 1884.19 & -0.99 & 6.03 & 23.87 \\
106 & {\color{red}0.0013} & 1581.38 & -0.71 & 6.13 & 17.86 \\
111 & {\color{red}0.0005} & 1427.92 & -0.51 & 6.68 & 15.23 \\
113 & {\color{red}0.0006} & 1564.59 & -0.61 & 6.92 & 18.00 \\
116 & {\color{red}> 0.0001} & 1778.92 & -0.85 & 7.29 & 24.86 \\
123 & {\color{red}> 0.0001} & 1702.39 & -0.75 & 8.22 & 25.08 \\
128 & {\color{red}0.0034} & 1524.91 & -0.50 & 8.96 & 20.09 \\
130 & {\color{red}0.0004} & 1775.9 & -0.77 & 9.28 & 29.06 \\
135 & {\color{red}> 0.0001} & 2259.56 & -1.24 & 10.12 & 50.58 \\
142 & {\color{red}0.0005} & 1819.58 & -0.76 & 11.42 & 35.51 \\
144 & {\color{red}> 0.0001} & 1710.17 & -0.63 & 11.82 & 31.35 \\
148 & {\color{red}> 0.0001} & 1162.24 & -0.11 & 12.66 & 15.47 \\
156 & {\color{red}> 0.0001} & 2103.29 & -0.95 & 14.54 & 55.36 \\
160 & {\color{red}0.0003} & 1929.59 & -0.78 & 15.58 & 49.14 \\
162 & {\color{red}0.0001} & 2097.63 & -0.9 & 16.13 & 58.13 \\
164 & {\color{red}> 0.0001} & 1981.77 & -0.79 & 16.69 & 53.52 \\
167 & {\color{red}> 0.0001} & 2593.04 & -1.31 & 17.58 & 94.08 \\
169 & {\color{red}> 0.0001} & 2285.37 & -1.03 & 18.2 & 74.64 \\
171 & {\color{red}> 0.0001} & 2260.4 & -1.02 & 18.84 & 77.16 \\
173 & {\color{red}> 0.0001} & 2537.07 & -1.25 & 19.5 & 98.52 \\
176 & {\color{red}> 0.0001} & 2566.57 & -1.21 & 20.54 & 100.13 \\
183 & {\color{red}0.0014} & 2210.85 & -0.85 & 23.17 & 79.57 \\
184 & {\color{red}0.0001} & 1651.4 & -0.42 & 23.58 & 47.83 \\
185 & {\color{red}> 0.0001} & 2572.07 & -1.20 & 23.99 & 116.21 \\
187 & {\color{red}> 0.0001} & 2518.29 & -1.11 & 24.83 & 111.00 \\
193 & {\color{red}> 0.0001} & 1995.25 & -0.66 & 27.54 & 76.25 \\
194 & {\color{red}> 0.0001} & 2467.91 & -1.06 & 28.02 & 119.13 \\
198 & {\color{red}> 0.0001} & 2235.46 & -0.80 & 30.03 & 97.20 \\ \hline
\end{longtable}
}
