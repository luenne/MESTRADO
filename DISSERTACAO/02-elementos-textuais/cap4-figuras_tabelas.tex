
\chapter{Figuras, Tabelas e outros elementos}\label{chap:fundamentacaoTeorica}
A seguir ilustra-se a forma de incluir figuras, tabelas, equações, siglas e símbolos no documento, obtendo indexação automática em suas respectivas listas.
A numeração sequencial de figuras, tabelas e equações ocorre de modo automático.
Referências cruzadas são obtidas através dos comandos \verb#\label{}# e \verb#\ref{}#.
Por exemplo, estou me referindo agora a introdução que corresponde ao  capítulo \ref{chap:introducao}. Todos os comandos utilizados para gerar os elementos desse texto estão nos respectivos arquivos dos capítulos. Não os colocaremos aqui para que o texto não fique demasiadamente extenso e não ficarmos repetitivos.

\section{Figuras}
\label{sec:figuras}

Abaixo é apresentado um exemplo de figura. A  figura \ref{fig:brasaouesc} aparece automaticamente na lista de figuras. Para uso avançado de imagens no LATEX, recomenda-se a consulta de literatura especializada.

\begin{figure}[!htb]
	\centering
	\caption{Brasão da UESC}
	\includegraphics[width=0.3\textwidth]{./04-figuras/brasaouesc}
	\fonte{\citeonline{teste:2014}}
	\label{fig:brasaouesc}
\end{figure}

\subsection{Figuras lado a lado}\label{figladoalado}
Para colocar figuras lado a lado com legendas diferentes utilizamos o pacote {\ttfamily subfigure}, declarado no preâmbulo. Porém o pacote abntex, no qual é baseado a classe, é incompatível com o pacote {\ttfamily subfigure}. Ao tentar usá-lo a compilação apresenta erros de incompatibilidade. 

Para contornar essa situação fiz algumas alterações necessárias no pacote e o renomeei como {\ttfamily subfigureppmgc}. Assim, caso você queira utilizar no seu texto figuras lado a lado com legendas diferentes, deve utilizar essa nova versão do  pacote, o qual se encontra na pasta {\ttfamily pacotes}. A classe já está configurada para utilizá-lo automaticamente.

Caso as figuras que fiquem lado a lado  utilizem apenas uma legenda não há necessidade do pacote, basta chamá-las normalmente no ambiente {\ttfamily figure}.

Abaixo segue um exemplo de como fica a utilização do pacote inserindo figuras lado a lado, com legendas diferentes.\\

\begin{figure}[!htb]
 \caption{Imagens comparando o estado do rio Cachoeira a dez anos atrás e atualmente} \label{fig:estadorc}
\subfigure[ Imagem do rio Cachoeira a dez anos atras\label{fig:rc-adezanos}]{
\includegraphics[width=7cm,height=6cm]{04-figuras/rc-adezanos}}
\subfigure[Imagem atual do rio Cachoeira\label{fig:rc-atualmente}]{
\includegraphics[width=7cm,height=6cm]{04-figuras/rc-atualmente}}
\fonte{Pimenta Blog.BR, no endereço em $<$http://www.pimenta.blog.br/$>$}
\end{figure}
\section{Quadros e Tabelas}
\label{sec:tabelas}

Também é apresentado o exemplo do quadro \ref{qua:comparabd} e da tabela \ref{tab:correlacao}, que aparece automaticamente na lista de quadros e tabelas.
Informações sobre a construção de tabelas no LATEX podem ser encontradas na literatura especializada e diversas apostilas encontradas livremente na internet  \cite{Lamport1986,Buerger1989,reginaldo,camargo}.

\input{./06-quadros/quacomparabd}
\newpage
Exemplos de tabelas:

\input{./05-tabelas/tabcorrelacao}

\input{./05-tabelas/tabteste}

\section{Equações}
\label{sec:equacoes}

A transformada de Laplace é dada na equação \ref{eq:laplace}, enquanto a equação \ref{eq:dft} apresenta a formulação da transformada discreta de Fourier bidimensional\footnote{Deve-se reparar na formatação esteticamente perfeita destas equações.}.

\begin{equation}
	X(s) = \int\limits_{t = -\infty}^{\infty} x(t) \, \text{e}^{-st} \, dt
	\label{eq:laplace}
\end{equation}

\begin{equation}
	F(u, v) = \sum_{m = 0}^{M - 1} \sum_{n = 0}^{N - 1} f(m, n) \exp \left[ -j 2 \pi \left( \frac{u m}{M} + \frac{v n}{N} \right) \right]
	\label{eq:dft}
\end{equation}


